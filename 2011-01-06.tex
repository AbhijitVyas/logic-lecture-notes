\section{VL von 6.~Januar 2011}

\subsection{Zusammenhang ist nicht FO-ausdrückbar}

Angenommen, es gibt Satz $\varphi\in FO(\tau)$, der Zusammenhang ausdrückt,
wobei $\tau=\set{E}$. Seien $c_1,c_2$ Konstantensymbole und für $n\geq 0$:

\[
  \psi_n = \NOT(\exists x_1,\dots,x_n (c_1=x_0 \AND c_2=x_1 \AND E(x_0,x_1) \AND \dots \AND E(x_{n-1},x_n)))
\]

$\psi_n$ drückt aus: \enquote{es gibt keinen Pfad der Länge $n$ von $c_1$ nach $c_2$.}

Setze $\Gamma=\set{\varphi}\cup\set{\psi_n|n\geq 0}$. Offensichtlich ist
$\Gamma$ unerfüllbar. Jede endliche Teilmenge $\Gamma_f\subseteq\Gamma$ ist
eber erfüllbar.

Sei $m$ maximal mit $\psi_m\in\Gamma_f$ Dann ist die folgende Struktur
ein Modell von $\Gamma_f$:

\begin{verbatim}
       E      E      E        E      E
    o ---> o ---> o ---> ... ---> o ---> o
   c_1    a_1    a_2             a_m    c_2
   
  `-------------------v--------------------´
                   m+1 Kanten
\end{verbatim}

Dies ist aber ein Widerspruch zum Kompaktheitstheorem.
\qed

\subsection{Beispiel Ehrenfeucht-Fraïssé}

Sei $\tau=\set{E}$, $E$ binäres Relationssymbol.

\begin{verbatim}
  Graphen von Carsten tikzen
\end{verbatim}

\begin{tabular}{ccc}
  Runde & Spoiler & Duplikator \\
  1 & $a_1$ & $b_2$ \\
  2 & $b_1$ & $a_4$ \\
  3 & $a_3$ & $b_1$ \\
\end{tabular}

\begin{align*}
  \delta_1 &= \begin{cases}
    a_1 \mapsto b_1 \\
    a_3 \mapsto b_3
  \end{cases} && \text{ist partieller Isomorphismus} \\
  \delta_2 &= \begin{cases}
    a_1 \mapsto b_2 \\
    a_2 \mapsto b_3
  \end{cases} && \text{ist partieller Isomorphismus} \\
  \delta_3 &= \begin{cases}
    a_1 \mapsto b_2 \\
    a_3 \mapsto b_3
  \end{cases} && \text{ist kein partieller Isomorphismus} \\
  \delta_4 &= \begin{cases}
    a_1 \mapsto b_1 \\
    a_3 \mapsto b_1
  \end{cases} && \text{ist kein partieller Isomorphismus}
\end{align*}

\subsubsection{Gewinnstrategien, Beispiel 1}

\begin{verbatim}
  Graphen von Carsten tikzen
\end{verbatim}

\begin{itemize}
  \item Duplikator gewinnt $G_0(\Afrak, \Bfrak)$.
  
  \item Duplikator hat Gewinnstrategie für $G_1(\Afrak,\Bfrak)$: (Graph links, Carsten, tikzen)\\
  Jeder Pfad definiert partiellen Isomorphismus, z.B. $\set{a_1\mapsto b_2}$.
  
  \item Spoiler hat Gewinnstrategie für $G_2(\Afrak,\Bfrak)$: (Graph rechts, Carsten, tikzen)\\
  Kein Pfad definiert partiellen Isomorphismus, z.B. $\set{a_1\mapsto b1, a_2\mapsto b_2}$.
\end{itemize}

\subsubsection{Gewinnstrategien, Beispiel 0}

\begin{itemize}  
  \item Duplikator hat Gewinnstrategie für $G_0(\Afrak,\Bfrak)$ und $G_1(\Afrak,\Bfrak)$
  
  \item Spoiler hat Gewinnstrategie für $G_2(\Afrak,\Bfrak)$ (Graph, Carsten...)
\end{itemize}

\subsection{Lemma: paarweise Äquivalenz}

Beweis per Induktion über Quantorenrank ($\qr$).

\begin{description}
  \item[IA:] $k=0$\\
  Da $\tau$ endlich und $m$ fixiert, gibt es nur endlich viele Atome.
  Jede Formel von Quantorenrank 0 ist Boolsche Kombination solcher Atome
  (aufgebaut mittels $\NOT, \AND, \OR$). Bis auf Äquivalenz gibt es nur
  endlich viele solcher Kombinationen: jede davon definiert eine der
  endlich vielen Boolschen Funktionen mit den Atomen als Aussagenvariablen.
  
  \item[IS:] $k>0$\\
  Jede Formel $\varphi$ mit $\qr(\varphi)=k$ ist Boolsche Kombination von
  \begin{itemize}
    \item Formeln mit Quantorenrank $< k$
    \item Formeln $Qy.\psi(\overline{x}, \overline{y})$ mit $Q\in\set{\exists,\forall}$ und $\qr(\varphi)=k-1$
  \end{itemize}
  Nach IV gibt es bis auf Äquivalenz nur endlich viele solcher Formeln,
  also auch nur endlich viele Boolsche Kombinationen.
\end{description}
\qed
