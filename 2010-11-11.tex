\section{VL vom 11. November 2010}

\subsection{Resolutionssatz Einheitsresolution}

\begin{description}
  \item[\enquote{$\Leftarrow$}]
  Wie im Resolutionssatz, denn es gilt $\ERes^*(M)\EQUIV M$.
  
  \item[\enquote{$\Rightarrow$}]
  Wir verwenden die Korrektheit unseres Polyzeit-Algorithmus für Erfüllbarkeit
  von Horn-Formeln. Setze in Analogie zu diesem Algorithmus:
  
  \begin{align}
    V^0     &:= \set{x| M(\varphi) \text{enthält} \set{x}} \\
    V^{i+1} &:= V^i \cup \set{x| \exists x_1,\dots,x_k \in V\ \text{mit}\ \set{\NOT x_1,\dots,\NOT x_k,x} \in M(\varphi)} \\
    V^*     &:= \bigcup_{i\geq 0} V^i \\
  \end{align}
  
  Mit der Korrektheit des Polyzeit-Algorithmus gilt
  \begin{itemize}
    \item[($*$)] $\varphi$ unerfüllbar gdw. es $x_1,\dots,x_k \in V^*$ gibt, mit
    $\set{\NOT x_1,\dots,\NOT x_k,x}\in M(\varphi)$.
  \end{itemize}
  
  Wir zeigen
  \begin{itemize}
    \item[($**$)] $x\in V^* \Rightarrow \set{x} \in \ERes^*(M(\varphi))$,
  \end{itemize}
  genauer gesagt: $x\in V' \Rightarrow \set{x} \in \ERes^*(M(\varphi))$ für
  alle $i\geq 0$.

  
  \begin{description}
    \item[IA] Trivial nach Definition $V^0$ und wegen $M(\varphi) \subseteq \ERes^*(M(\varphi))$
    \item[IS] Sei $x \in V^{i+1}$. Wenn $x\in V^i$ verwende IA.
    
    Sonst gibt es $x_1,\dots,x_k \in V^i$ mit $\set{\NOT x_1,\dots,\NOT x_k,x} \in M(\varphi)$.
    Wegen folgen der Ableitung von $\set{x}$ mittels Einheitsresolution gilt $\set{x}\in \ERes^*(M(\varphi))$.
    \begin{verbatim}
      \set{\NOT x_1,\dots,x_k,x}   \set{x_1}   \set{x_k}
                \                   /           /
              \set{\NOT x_2,\dots,x_k,x}       /
                                 .            /
                                  .          /
                                   .        /
                                    \set{x}
    \end{verbatim}
    
    Sei $\varphi$ unerfüllbar. Mit ($*$) und ($**$) gibt es
    $\set{\NOT x_1,\dots,\NOT x_k}\in M(\varphi)$ mit
    $\set{x_1},\dots,\set{x_k}\in\Res^*(M(\varphi))$.
    Folgende Ableitung mittels Einheitsresolution zeigt
    $\square\in\ERes^*(M(\varphi))$.
    \begin{verbatim}
      \set{\NOT x_1,\dots,x_k}   \set{x_1}   \set{x_k}
                \                 /           /
              \set{\NOT x_2,\dots,x_k}       /
                                 .          /
                                  .        /
                                   .      /
                                   \square
    \end{verbatim}
    \qed
  \end{description}
\end{description}
