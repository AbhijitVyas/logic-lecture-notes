\section{VL vom 11. November 2010}

\subsection{Resolutionssatz Einheitsresolution}

\begin{description}
  \item[\enquote{$\Leftarrow$}]
  Wie im Resolutionssatz, denn es gilt $\ERes^*(M)\EQUIV M$.
  
  \item[\enquote{$\Rightarrow$}]
  Wir verwenden die Korrektheit unseres Polyzeit-Algorithmus für Erfüllbarkeit
  von Horn-Formeln. Setze in Analogie zu diesem Algorithmus:
  
  \begin{align}
    V^0     &:= \set{x| M(\varphi) \text{enthält} \set{x}} \\
    V^{i+1} &:= V^i \cup \set{x| \exists x_1,\dots,x_k \in V\ \text{mit}\ \set{\NOT x_1,\dots,\NOT x_k,x} \in M(\varphi)} \\
    V^*     &:= \bigcup_{i\geq 0} V^i \\
  \end{align}
  
  Mit der Korrektheit des Polyzeit-Algorithmus gilt
  \begin{itemize}
    \item[($*$)] $\varphi$ unerfüllbar gdw. es $x_1,\dots,x_k \in V^*$ gibt, mit
    $\set{\NOT x_1,\dots,\NOT x_k,x}\in M(\varphi)$.
  \end{itemize}
  
  Wir zeigen
  \begin{itemize}
    \item[($**$)] $x\in V^* \Rightarrow \set{x} \in \ERes^*(M(\varphi))$,
  \end{itemize}
  genauer gesagt: $x\in V' \Rightarrow \set{x} \in \ERes^*(M(\varphi))$ für
  alle $i\geq 0$.

  
  \begin{description}
    \item[IA]
    Trivial nach Definition $V^0$ und wegen $M(\varphi) \subseteq \ERes^*(M(\varphi))$
    
    \item[IS]
    Sei $x \in V^{i+1}$. Wenn $x\in V^i$ verwende IA.
    
    Sonst gibt es $x_1,\dots,x_k \in V^i$ mit $\set{\NOT x_1,\dots,\NOT x_k,x}\in M(\varphi)$.
    Wegen folgen der Ableitung von $\set{x}$ mittels Einheitsresolution gilt
    $\set{x}\in \ERes^*(M(\varphi))$.
    
    \begin{center}
      \begin{tikzpicture}
        \node (a) {$\set{\NOT x_1,\dots,x_k,x}$};
        \node[right=of a] (b) {$\set{x_1}$};
        \node[right=of b] (c) {\dots};
        \node[right=of c] (d) {$\set{x_k}$};
        \node[xshift=4em,below=of a] (e) {$\set{\NOT x_2,\dots,x_k,x}$};
        \node[xshift=6em,below=of e] (f) {$\set{x}$};
        \draw (a)--(e) (b)--(e) (d)--(f);
        \draw[thick,loosely dotted] (e)--(f);
      \end{tikzpicture}
    \end{center}
    
    Sei $\varphi$ unerfüllbar. Mit ($*$) und ($**$) gibt es
    $\set{\NOT x_1,\dots,\NOT x_k}\in M(\varphi)$ mit
    $\set{x_1},\dots,\set{x_k}\in\Res^*(M(\varphi))$.
    Folgende Ableitung mittels Einheitsresolution zeigt
    $\square\in\ERes^*(M(\varphi))$.
    
    \begin{center}
      \begin{tikzpicture}
        \node (a) {$\set{\NOT x_1,\dots,x_k}$};
        \node[right=of a] (b) {$\set{x_1}$};
        \node[right=of b] (c) {\dots};
        \node[right=of c] (d) {$\set{x_k}$};
        \node[xshift=4em,below=of a] (e) {$\set{\NOT x_2,\dots,x_k}$};
        \node[xshift=6em,below=of e] (f) {$\square$};
        \draw (a)--(e) (b)--(e) (d)--(f);
        \draw[thick,loosely dotted] (e)--(f);
      \end{tikzpicture}
    \end{center}
    \qed
  \end{description}
\end{description}

\subsection{Beispiel Herleitbarkeit im Hilbert-Kalkül}

Herleitung $x \IMPL x$.

Herleitbare Formeln:
\begin{enumerate}[label=(\alph*)]
  \item $x \IMPL ((y\IMPL x)\IMPL x)$                                           \hfill Instanz Axiom 1
  \item $(x\IMPL((y\IMPL x)\IMPL x)\IMPL((x\IMPL(y\IMPL x))\IMPL(x\IMPL x)))$   \hfill Instanz Axiom 2
  \item $(x\IMPL(y\IMPL x))\IMPL (x\IMPL x))$                                   \hfill MP + (a) + (b)
  \item $x\IMPL(y\IMPL x)$                                                      \hfill Instanz Axiom 1
  \item $x\IMPL x$                                                              \hfill MP + (c) + (d)
\end{enumerate}

\subsection{Kompaktheitssatz}

Wir bewisen Punkt 1 und dort nur \enquote{$\Leftarrow$}, denn \enquote{$\Rightarrow$} ist trivial.

Wir nennen eine Menge $\Gamma\subseteq AL$ \textit{Limit-erfüllbar}, wenn jede
endliche Teilmenge von $\Gamma$ erfüllbar ist. Sei also $\Gamma$ Limit-erfüllbar.
Da $\Var$ abzählbar, gibt es offensichtlich Aufzählung
\[
  \varphi_1,\varphi_2,\varphi_3,\dots \quad\text{von AL.}
\]

Wir konstruieren Folge von Limit-erfüllbaren Formelmengen
\[
  \Gamma = \Gamma_0 \subseteq \Gamma_1 \subseteq \Gamma_2 \subseteq \dots
\]

wie folgt: im $\Gamma_{i+1}$ zu definieren, beobachte zunächst
$\Gamma_i\cup\set{\varphi_{i+1}}$ oder $\Gamma_i\cup\set{\NOT\varphi_{i+1}}$
ist Limit-erfüllbar.

\textbf{Beweis:} Nimm im Gegenteil an, dass es endliche
$\Delta,\Delta'\subseteq\Gamma_i$, so dass $\Delta\cup\set{\varphi_{i+1}}$ und
$\Delta'\cup\set{\NOT\varphi_{i+1}}$ unerfüllbar. Dann ist
$\Delta\cup\Delta'\subseteq\Gamma_i$ unerfüllbar, im Widerspruch zur
Limit-Erfüllbarkeit von $\Gamma_i$.

Setze
\[
  \Gamma_{i+1} = \begin{cases}
    \Gamma_i \cup \set{\varphi_{i+1}} \quad\text{wenn $\Gamma_i\cup\set{\varphi_{i+1}}$ Limit-erfüllbar} \\
    \Gamma_i \cup \set{\varphi_{i+1}} \quad\text{sonst.} \\
  \end{cases}
\]

\textbf{Behauptung:} Angenommen, das ist nicht der Fall. Dann gibt es endliches
$\Delta\subseteq\Gamma_\omega$, das unerfüllbar ist. Da $\Delta$ endlich, gibt
es $i\geq 0$ mit $\Delta\subseteq\Gamma_i$, im Widerspruch zur
Limit-Erfüllbarkeit von $\Gamma_i$.

Definiere Belegung V wie folgt:
\[
  V(x) = \begin{cases}
    1 \quad\text{wenn $x \in \Gamma_\omega$}\\
    0 \quad\text{sonst}
  \end{cases}
\]

Wie zeigen, dass $V\models\Gamma_\omega$, also auch $V\models\Gamma_\omega$, damit $\Gamma$ erfüllbar.
Per Induktion über die Struktur von $\varphi$:
\[
  V \models \varphi\ \text{gdw.}\ \varphi \in \Gamma_\omega\ \text{für alle}\ \varphi \in AL
\]

\begin{enumerate}
  \item $\varphi$ ist Variable.\par
  Folgt aus Definition von V.
  
  \item $\varphi \not= \psi$.\par
  Dann $V\models\varphi$, gdw. $V\not\models\psi$ gdw. $\psi\not\in\Gamma_\omega$ gdw. $\varphi\in\Gamma_\omega$.
  
  \item $\varphi = \psi\AND\vartheta$.\par
  \begin{itemize}
    \item[\enquote{$\Rightarrow$}]
      $V\models\varphi \IMPL V\models\psi$ und $V\models\vartheta$\\
      $V\models\varphi \IMPL \psi\in\Gamma_\omega$ und $\vartheta\in\Gamma_\omega$
                                                    
    Nach Konstruktion gilt $\psi\AND\vartheta\in\Gamma_\omega$ oder
    $\NOT(\psi \AND \vartheta) \in \Gamma_\omega$. Nun ist aber
    $\set{\psi, \vartheta, \NOT(\psi \AND \vartheta)}$
    unerfüllbar (im Widerspruch zu (*)), also $\psi\AND\vartheta\in\Gamma_\omega$.
    
    \item[\enquote{$\Leftarrow$}]
    Wenn $\psi\AND\vartheta\in\Gamma_\omega$, dann $\psi,\vartheta\in\Gamma_\omega$
    (sonst $\NOT\psi\in\Gamma_\omega$ und dann wäre
    $\set{\psi\AND\vartheta, \NOT\psi}$ unerfüllbar, ebenso für $\vartheta$).
    Nach IV $V\models\psi$ und $V\models\vartheta$, also $V\models\psi\AND\vartheta$.
    \qed
  \end{itemize}
\end{enumerate}

\subsubsection{Anwendung: 4-Färbbarkeit}

\begin{description}
  \item[\enquote{$\Leftarrow$}] ist trivial.
  \item[\enquote{$\Rightarrow$}] Sei $G=(V,E)$ Graph, für den jeder endliche
  Teilgraph 4-färbbar ist.

  Definiere Formelmenge:
  \[
    \Gamma = \set{x_{v_1} \OR x_{v_2} \OR x_{v_3} \OR x_{v_4} \mid v \in V}
      \cup \set{\ANDop_{i\in\set{1,2,3,4}} \NOT (x_{v_i} \AND x_{v_i'}) \mid (v,v')\in E }
  \]

  Wegen des 4-Farben-Satzes auf endlichen Graphen ist jede endliche Teilmenge
  von $\Gamma$ erfüllbar. Wegen Kompaktheit ist $\Gamma$ erfüllbar, also ist
  $(V,E)$ 4-färbbar.
  \qed
\end{description}

