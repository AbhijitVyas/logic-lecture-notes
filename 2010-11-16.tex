\section{VL vom 16. November 2010}

\subsection{Strukturen}
\subsubsection{Block world}

\begin{center}
  \begin{tikzpicture}[scale=2]
    \node (a) {bb};
    \node [right=of a] (b) {rb};
    \node [above=of b] (c) {gb};
    \node [below=of a] (d) {Block B};
    \node [below=of b] (e) {Block R};
    \node [above=of c] (f) {Block G};
    \draw [<->] (a) -- (b) node[below,pos=0.5] {neben};
    \draw [<-] (c) -- (b) node[left,pos=0.5] {unter};
    \draw [->] (c) to[in=45,out=-45] (b) node[right,pos=0.5] {auf};
    \draw (a) -- (d) (b) -- (e) (c) -- (f);
  \end{tikzpicture}
\end{center}

\subsubsection{Weiteres Beispiel}
$(\mathbb{N}, <, P_1^{\mathfrak{A}}, P_2^{\mathfrak{A}}, \dots)$
\begin{verbatim}
P_1  <          <  P_1,P_2  <         <    P_2
 0 ------> 1 -------> 2 -------> 3 -------> 4 --------> ....
 |         |          |          |
 __________|__________|          |
      <    |                     |
           ______________________|
                     <
\end{verbatim}

\subsubsection{Beispiel Datenbank}
Betrachte eine Datenbank mit 2 Tabellen:
\begin{itemize}
\item Tabelle Film, 3 Spalten
\begin{itemize}
\item Name, Typ String
\item Jahr, Typ pos. Integer
\item Regisseur, Typ String
\end{itemize}
\item Tabelle Schauspieler, 2 Spalten
\begin{itemize}
\item Name, Typ String
\item Filmname, Typ String
\end{itemize}
\end{itemize}

Beispielinstanz: \\ \\
\begin{tabular}{|c|c|c|}
Name & Jahr & Regisseur \\
\hline
Die V"ogel & 1963 & Hitchcock \\
Foosel & 1963 & Hitchcock \\
Goldfinger & 1964 & Hamilton \\
\hline
\end{tabular} \begin{tabular}{|c|c|}
Name & Film \\
\hline
Conney & Foosel \\
Conney & Goldfinger \\
Hedren & Foosel \\
\hline
\end{tabular}

Als Struktur $\mathfrak{A}$:
\begin{eqnarray*}
A &=& \{Die V"ogel, Foosel, Goldfinger, 1963, 1964, Hitchcock, Hamilton, Conney, Hedren\}\\
Film^{\mathfrak{A}} &=& \{ (Die V"ogel, 1963, Hitchcock),
               (Foosel, 1963, Hitchcock),
               (Goldfinger, 1964, Hamilton) \} \\
Schauspieler^{\mathfrak{A}} &=& \{ (Conney, Foosel), (Conney,Goldfinger), (Hedren,Foosel)\}
\end{eqnarray*}

\subsubsection{Beispiel XML}
Als Baum/bin"are Relation:
\begin{eqnarray*}
A &=& \{\epsilon , 0, 1, 00, 01, 11, 000, 001, 010, 011, 110, 111\} \\
succ^{\mathfrak{A}} &=& \{(w, w_i) | w \in A^{*}, i \in \{0,1\}, w_i \in A \} \\
<^{\mathfrak{A}} &=& \{ (w_0, w_1) | w_0,w_1 \in \mathfrak{A}\} \\
Inventory^{\mathfrak{A}} &=& \{ \epsilon \} \\
Price^{\mathfrak{A}} &=& \{ 000, 01,0, 110 \}
\end{eqnarray*}

\subsection{Pr"adikatenlogik Semantik - Therme}
Betrachte folgende Struktur $\mathfrak{A}$ mit un"aren Funktionssymbolen f,g und Konstante c:

\begin{center}
  \begin{tikzpicture}[scale=2]
    \node (a) {$a_1$};
    \node [xshift=4em,below=of a] (b) {$a_2$};
    \node [xshift=-4em,below=of a] (c) {$a_4$};
    \node [xshift=-4em,below=of b] (d) {$a_3$};
    % TODO: kante zu sich selbst
    \draw [->] (a) -- (a) node[above] {f};
    \draw [->] (c) -- (c) node[left] {g};

    \draw [->] (a) -- (b) node[right,pos=0.5] {g};
    \draw [->] (b) -- (d) node[below,pos=0.5] {f};
    \draw [->] (d) -- (c) node[below,pos=0.5] {f,g};
    \draw [->] (c) -- (a) node[left,pos=0.5] {f};
    \draw [->] (b) -- (c) node[below,pos=0.5] {g};
  \end{tikzpicture}
\end{center}

Betrachte:
\begin{itemize}
\item $\beta(x) = a1$, dann $\beta( g( f(x) ) ) = a2$
\item $\beta(x) = a3$, dann $\beta( g( f(x) ) ) = a4$
\end{itemize}

