\documentclass[10pt,a4paper]{article}
\usepackage[utf8]{inputenc}
\usepackage[T1]{fontenc}
\usepackage[ngerman]{babel}
\usepackage[sc,osf]{mathpazo}
\usepackage[german=guillemets]{csquotes}
\usepackage[fleqn]{amsmath}
\usepackage{amssymb,parskip,xspace,textcomp,latexsym,stmaryrd,enumitem,url,booktabs,array}
\usepackage[margin=3cm]{geometry}
\usepackage{tikz}
\usetikzlibrary{positioning}

\DeclareMathOperator{\AND}{\wedge}
\DeclareMathOperator{\OR}{\vee}
\DeclareMathOperator{\NOT}{\neg}
\DeclareMathOperator{\IMPL}{\rightarrow}
\DeclareMathOperator{\EQUIV}{\equiv}
\DeclareMathOperator*{\ANDop}{\bigwedge}
\DeclareMathOperator*{\ORop}{\bigvee}
\DeclareMathOperator{\UNION}{\cup}
\DeclareMathOperator{\SUBSET}{\subset}

\DeclareMathOperator{\Var}{\text{\textsf{Var}}}
\DeclareMathOperator{\Res}{\text{\textsf{Res}}}
\DeclareMathOperator{\ERes}{\text{\textsf{ERes}}}
\DeclareMathOperator{\ans}{\text{\textsf{ans}}}
\DeclareMathOperator{\qr}{\text{\textsf{qr}}} % Quantorenrank

\DeclareMathOperator{\cl}{\text{\textsf{cl}}}% closure, Abschluss
\DeclareMathOperator{\Taut}{\text{\textsf{Taut}}}% Tautologie
\DeclareMathOperator{\Th}{\text{\textsf{Th}}}% Theorie...
\DeclareMathOperator{\Sig}{\text{\textsf{Sig}}}% Signatur
\DeclareMathOperator{\dom}{\text{\textsf{dom}}}% Definitionsbereich
\DeclareMathOperator{\ran}{\text{\textsf{ran}}}% Wertebereich
\def\Afrak{\ensuremath{\mathfrak{A}}}
\def\Bfrak{\ensuremath{\mathfrak{B}}} % I'm lazy
\def\Hfrak{\ensuremath{\mathfrak{H}}} % Herbrand-Struktur


\def\roem#1{\ensuremath{\text{#1}}\xspace}
\def\set#1{\ensuremath{\left\{#1\right\}}}
\def\I{\roem{I}}
\def\II{\roem{II}}
\def\III{\roem{III}}
\def\qed{\strut\\\null\hfill$\square$}
\def\QED{\strur\\\null\hfill$\blacksquare$}

\begin{document}

\section*{Logik}

\begin{itemize}
  \item Webseite der Vorlesung:\\
  \url{http://www.informatik.uni-bremen.de/tdki/lehre/ws10/logik/}
  \item Mitschriften online verfügbar:\\
  \url{https://github.com/daniel86/logic-lecture-notes}
\end{itemize}

\section{VL vom 26.~Oktober 2010}

Wir verwenden die Variablen 

\begin{align}
  x_{ij}^\alpha \quad\text{mit}\quad \alpha \in \set{M,S,K}, i \in \set{\I,\II,\III}, j \in \set{a,b}
\end{align}

Zum Beispiel repräsentiert die Variable $x_{\II a}^M$ die Aussage
\enquote{Müller unterrichtet in Stunde \II Klasse $a$}.

Wir modellieren nun das Problem wie folgt:

\begin{itemize}
  \item Jede Stunde wurd von einem passendem Lehrer unterrichtet
  \begin{align}
    \varphi_1 &=
      \ANDop_{ij \in\set{\I a,\II a,\III b}} \biggl(x_{ij}^M \OR x_{ij}^K\biggr) \AND
      \ANDop_{ij \in\set{\I b,\II a,\II b}}  \biggl(x_{ij}^S \OR x_{ij}^K\biggr)
  \end{align}
  
  \item Jede Stunde wird von höchstens einem Lehrer unterrichtet:
  \begin{align}
    \varphi_2 &=
      \ANDop_{i\in\set{\I,\II,\III}} \ANDop_{j\in\set{a,b}}
        \NOT \biggl(x_{ij}^M \AND x_{ij}^S\biggr) \AND
        \NOT \biggl(x_{ij}^M \OR x_{ij}^K\biggr) \AND
        \NOT \biggl(x_{ij}^S \OR x_{ij}^K\biggr)
  \end{align}
  
  \item Jeder Lehrer unterrichtet mindesten zwei Stunden:
  \begin{align}
    \varphi_3 &=
      \ANDop_{\alpha\in\set{M,K,S}} \ORop_{\substack{i,i'\in\set{\I,\II,\III}\\ j,j'\in\set{a,b}\\ ij\not=i'j'}}
        \biggl(x_{ij}^\alpha \AND x_{i'j'}^\alpha\biggr)
  \end{align}
  
  \item Ein Lehrer kann nur eine Klasse zur Zeit unterrichten
  \begin{align}
    \varphi_4 &=
      \ANDop_{\alpha\in\set{M,K,S}} \ANDop_{i\in\set{\I,\II,\III}}
        \NOT \biggl(x_{ia}^\alpha \AND x_{ib}^\alpha\biggr)
  \end{align}
\end{itemize}

Man sieht nun leicht, dass die möglichen Lösungen für das Zeitplanungsproblem
ganau dem Belegungen $V$ entsprechen, die
$\varphi_1 \AND \varphi_2 \AND \varphi_3 \AND \varphi_4$ erfüllen.

\section{VL vom 28.~Oktober 2010}

\subsection{Ersetzungslemma}

\begin{description}
  \item[IA:]
  Wenn $\vartheta$ atomar ist, muss $\vartheta = \varphi$. Dann $\vartheta'=\psi$,
  also $\vartheta\EQUIV\vartheta'$ wegen $\psi\EQUIV\varphi$.

  \item[IS:]
  Wenn $\vartheta=\varphi$, argumentiere wir im IA. Sonst unterscheide drei Fälle:

  \begin{enumerate}
    \item $\vartheta = \NOT \vartheta_1$\\
    $\vartheta'$ hat die Form $\NOT\vartheta'_1$ (wobei sich $\vartheta'_1$ aus
    $\vartheta_1$ ergibt durch Ersetzen von $\varphi$ durch $\psi$. Nach IV gilt
    $\vartheta_1\EQUIV\vartheta'_1$, nach Semantik von \enquote{$\NOT$} also
    $\vartheta\EQUIV\vartheta'$.
    
    \item $\vartheta = \vartheta_1\OR\vartheta_2$\\
    $\varphi$ wird entweder in $\vartheta_1$ ider in $\vartheta_2$ durch $\psi$
    ersetzt. Wir betrachten nur den ersten Fall: dann
    $\vartheta'=\vartheta'_1\OR\vartheta_2$, also $\vartheta\EQUIV\vartheta'$
    
    \item $\vartheta = \vartheta_1\AND\vartheta_2$\\
    Analog zu 2.
  \end{enumerate}
  \qed
\end{description}


\subsection{Funktionale Vollständigkeit}

Die Funktionen in $\mathcal{B}^0$ werden durch die Formeln $0,1$ dargestellt.
Sei $n>0$ und $f\in\mathcal{B}^n$.

Für jedes $x\in Var$ sei $x^1=x$ und $x^0=\NOT x$. Für jedes Tupel
$f=(w_1,\dots,w_n)\in\set{0,1}^n$ sei $\varphi_f=x_1^{w_1}\AND\dots\AND x_n^{w_n}$.

\textbf{Definiere}
\begin{align}
  \varphi_f = \ORop_{\substack{f\in\set{0,1}^n\\ f(t) = 1}} \varphi_t
\end{align}

\textbf{Behauptung:} $f_{\varphi_f} = f$

Wenn $f(t) = 1$, dann ist $\varphi_t$ ein Disjunkt von $\varphi_f$. Also
$f_{\varphi_f}(t) = V_t(\varphi_f)=1$. Wenn umgekehrt $f_{\varphi_f}(t) = 1$,
dann $V_t(\varphi_f)=1$, also ist $\varphi_t$ Disjunkt von $\varphi_f$. Nach
Definition von $\varphi_f$ also $f(t)=1$. \qed


\subsection{Normalformen}

Sei $\varphi$ eine Formel. Äquivalente Formel in DNF ($\OR\AND$):

Konstruiere erst $f_\varphi$ mittels Wahrheitstafel, dann $\varphi_{f_\varphi}$
wie im vorigen Beweis. Das Resultat ist offensichtlich in DNF und die
Konstrktion ist effektiv.

Aus der effektiven Konstruierbarkeit der DNF folgt auch die der KNF ($\AND\OR$).

\begin{align}
  \varphi &\EQUIV \NOT\NOT \varphi                                 &&\text{DNF!}\\
          &\EQUIV \NOT\ORop_{i=1}^n \ANDop_{j=1}^{m_i} \ell_{i,j}  &&\text{de Morgan}\\
          &\EQUIV \ANDop_{i=1}^n \NOT\ANDop_{j=1}^{m_1} \ell_{i,j} &&\text{de Morgan}\\
          &\EQUIV \ANDop_{i=1}^n \ORop_{j=1}^{m_1} \NOT\ell_{i,j}  &&\text{KNF}
\end{align}
\qed

\section{VL vom 02. November 2010}

\subsection{Erfüllbarkeit ist in NP}

Sei $\phi$ Eingabeformel und sein $n = |\Var(\phi)|$.

Eine nicht-deterministische Touringmaschine kann mit $n$ nicht-determinischen
Übergängen eine Belegung für $\phi$ auf das Band schreiben. Danach prüft sie
deterministisch in Polynomialzeit, ob die geschriebene Belegung $\phi$ erfüllt.

Sie Aktzeptiert, wenn das der Fall ist und verwirft sonst. Offenbar löst die
Maschnine das Erfüllbarkeitsproblem in Polynomialzeit.

Erfüllbarkeit NP-hart $\leadsto$ VL Komplexitätstheorie (Cook's Theorem).

Gültigkeit: Reduktion auf Unerfüllbakeit 

\subsection{Korrektheit + Polynomialzeit für Horn-Formeln}

Offensichtlich terminiert der Algorithmus auf jeder Eingabe $\phi$ nach max.
$|\Var(\phi)|$ durchläufen der while-Schleife, also in Polynomialzeit.

Angenommen, der Algorithmus antowrtet \enquote{erfüllbar}. Dann gilt für die
konstruierte Menge $V$:

\begin{enumerate}
  \item Wenn $x_1\AND\dots\AND x_n \IMPL x$ Konjunkt von $\phi$ und $\set{x_1,\dots,x_n} \subseteq V$, dann $x \in V$.
  \item Wenn $x_1\AND\dots\AND x_n \IMPL 0$ Konjunkt von $\phi$, dann $\set{x_1,\dots,x_n} \not\subseteq V$.
\end{enumerate}

Also erfüllt $V$ (betrachtet als Belegung) $\phi$ und $\phi$ ist erfüllbar.

Angenommen, $\phi$ ist erfüllbar. Man zeigt leicht per Induktion über die Anzahl der Schleifendurchläufe:

\begin{itemize}
  \item[$*$] Wenn $x \in V$, dann $\hat{V}=1$ für alle Modell $\hat{V}$ von $\phi$.
\end{itemize}

Sei $x_1\AND\dots\AND x_n \IMPL 0$ Konjunkt von $\phi$. Es gilt
$\set{x_1,\dots,x_n} \not\subseteq V$, denn $\phi$ besitzt Modell $\hat{V}$ und
wenn $\set{x_1,\dots,x_n}\subseteq V$ ist mit $(*)$ auch
$\set{x_1,\dots,x_n} \subseteq \hat{V}$, im Widerspruch dazu dass $\hat{V}$ Modell
von $\phi$ ist. \qed

\subsubsection{Beispiel}

\[
  V=\set{\text{Regen}, \text{Schnee}} \cup \set{\text{Niederschlag}, \underline{\text{Temp}>0}, \underline{\text{Temp}\leq 0}}
\]

\section{VL vom 09. November 2010}

\subsection{Beweis Resolutionslemma}

\begin{itemize}
  \item $V \models  M \cup \set{C} \Rightarrow V \models  M$ trivial.
  \item Es gelte $V\models M$. Ferner sei $C=(C_1\backslash \set{\ell}) \cup
  (C_2\backslash \set{\overline{\ell}})$ mit $C_1,C_2 \in M$. Unterscheide zwei
  Fälle:
  \begin{itemize}
    \item $V(\ell) = 1$. Wegen $V\models C_2$ dann auch $V\models C_2\backslash \set{\overline{\ell}}$.
    \item $V(\ell) = 0$. Wegen $V\models C_1$ dann auch $V\models C_1\backslash \set{\overline{\ell}}$.
  \end{itemize}
  In beiden Fällen also $V\models C$.\qed
\end{itemize}

\subsubsection{Beispiel-Resolution}

\begin{align}
  M(\varphi) &= \set{\set{x_1}, \set{\NOT x_1, x_2}, \set{\NOT x_2, x_3}, \set{\NOT x_3}} = M \\
  \Res^0(M)   &= M \\
  \Res^1(M)   &= \Res^0(M) \cup \set{\set{x_2}, \set{\NOT x_1, x_3}, \set{\NOT x_2}} \\
  \Res^2(M)   &= \Res^1(M) \cup \set{\set{x_3}, \set{\NOT x_1}, \square} \\
  \Res^3(M)   &= \Res^2(M) = \Res^*(M)
\end{align}

\subsection{Beweis Resolutionssatz}

\begin{itemize}
  \item[$\Leftarrow$] \textbf{Korrektheit}\par
  Da $\square \in \Res^*(M)$, ist $\Res^*(M)$ unerfüllbar. Es genügt also zu zeigen, dass $\Res^*(M)\EQUIV M$.
  Mittels Resolutionslemma und per Induktion über $i$ zeigt man leicht, dass $M\EQUIV \Res^i(M) \forall i \geq 0$.
  Da sich über den endlich vielen Literalen in $M$ nur endlich viele Klauseln bilden lassen, ist $\Res^*(M)$ endlich, also $\Res^*(M) = \Res^i(M)$ für ein $i\geq 0$, damit $M\EQUIV \Res^*(M)$.
  
  \item[$\Rightarrow$] \textbf{Vollständigkeit}\par
  Wir zeigen
  \[
    M\ \text{unerfüllbar} \IMPL \square \in \Res^*(M)
  \]
  per Induktion über $|\Var(M)|$.
  \begin{description}
    \item[IA] $\Var(M) \not= \emptyset$. Dann $M=\emptyset$ oder $M=\set{\square}$.
    Da $\emptyset$ erfüllbar, muss $M=\set{\square}$ sein, also $\square\in M \subseteq \Res^*(M)$.
    
    \item[IS] Wähle $x\in \Var(M)$, konstruiere zwei Klauselmengen:
    \begin{align}
      K^+ &:= \set{C \backslash \set{\NOT x} | x \not\in C \in M} \\
      K^- &:= \set{C \backslash \set{x} | \NOT x \not\in C \in M} 
    \end{align}
    Intuitiv entspricht $K^+$ dem Fall $V(x) = 1$: alle Klauseln $C$ mit $x \in C$
    sind erfüllt und wurden gestrichen, aus den verbliebenen Klauseln kann
    $\NOT x$ gestrichen werden (wenn vorhanden), denn $\NOT x$ kann die Klausel
    nicht wahr machen.
    
    Wir zeigen:
    \begin{enumerate}
      \item $K^+$ und $K^-$ sind unerfüllbar
      \item $\square \in \Res^*(M)$ oder $\set{\NOT x} \in \Res^*(M)$
      \item $\square \in \Res^*(M)$ oder $\set{x} \in \Res^*(M)$
    \end{enumerate}
    
    \item[(1)] Angenommen, $K^+$ ist erfüllbar und $V\models K^+$.
    Erweitere $V$ durch $V(x)=1$. Man prüft leicht, dass $V\models M$. $\lightning$
    
    Unerfüllbarkeit $K^-$ analog.
    
    \item[(2)] Weil $K^+$ unerfüllbar liefert IV $\square\in\Res^*(K^+)$.
    Also gibt es Klauseln $C_1,\dots,C_m$, so dass $C_m=\square$ und für $1\leq i\leq m$ gilt
    \begin{enumerate}
      \item $C_i \in K^+$ oder
      \item $C_i$ ist Resolvente von $C_j,C_k$ für je gewisse $j,k < i$.
    \end{enumerate}

    \textbf{Fall 1:} alle Klauseln $C_i$ der Form (a) sind auch in $M$ (in keiner der
    \enquote{Originalklauseln} kam $\NOT x$ vor). Dann prüft man leicht, dass
    $C_1,\dots,C_m \in\Res^*(M)$, also $\square\in\Res^*(M)$.
    
    \textbf{Fall 2:} Für mind. ein $C_i$ der Form (a) ist $C_i\cup\set{\NOT x}\in M$.
    Wir erhalten duch Wiedereinfügen von $\NOT x$ eine Folge von Klauseln
    $C'_1,\dots,C'_m \in \Res^*(M)$, die beweist, dass $\set{\NOT x}\in\Res^*(M)$.
    
    \begin{center}
      \begin{tikzpicture}
        \node (j) at (-1,1) {$C_j$};
        \node (k) at (1,1) {$C_k$};
        \node (i) at (0,0) {$C_i$};
        \draw (j)--(i) (k)--(i);
        
        \node at (3,0.5) {$\Rightarrow$};
        \node (j) at (5,1) {$C_j\cup\set{\NOT x}$};
        \node (k) at (9,1) {$C_k\cup\set{\NOT x}$};
        \node (i) at (7,0) {$C_i\cup\set{\NOT x}$};
        \draw (j)--(i) (k)--(i);
      \end{tikzpicture}
    \end{center}
    \item[(3)] Analog zu (2) unter Verwendung von $K^-$.
  \end{description}

  Aus (2) und (3) folgt, dass $\square\in\Res^*(M)$ oder $\set{x},\set{\NOT x}\in\Res^*(M)$. Mit
  \begin{center}
    \begin{tikzpicture}
      \node (j) at (-1,1) {$\set{\NOT x}$};
      \node (k) at (1,1) {$\set{x}$};
      \node (i) at (0,0) {$\square$};
      \draw (j)--(i) (k)--(i);
    \end{tikzpicture}
  \end{center}
  dass auch $\square\in\Res^*(M)$.\qed
\end{itemize}

\subsection{Beispiel: Einheitsresolvente}

\begin{center}
  \begin{tikzpicture}
    \node             (a) {$\set{\NOT x_1, \NOT x_2, \NOT x_3, x_4}$};
    \node[right=of a] (b) {$\set{x_1}$};
    \node[right=of b] (c) {$\set{x_2}$};
    \node[right=of c] (d) {$\set{x_3}$};
    \node[right=of d] (e) {$\set{\NOT x_3, \NOT x_4}$};
    \node[xshift=4em,below=of a] (f) {$\set{\NOT x_2, \NOT x_3, x_4}$};
    \node[xshift=5em,below=of f] (g) {$\set{\NOT x_3, x_4}$};
    \node[xshift=5em,below=of g] (h) {$\set{x_4}$};
    \node[xshift=5em,below=of h] (i) {$\set{\NOT x_3}$};
    \node[xshift=6em,below=of i] (j) {$\square$};
    
    \draw (a)--(f) (b)--(f)
      (c)--(g) (f)--(g)
      (d)--(h) (g)--(h)
      (e)--(i) (h)--(i)
      (d) to[in=45,out=-45] (j) (i)--(j);
  \end{tikzpicture}
\end{center}

\section{VL vom 11. November 2010}



\section{VL vom 16. November 2010}

\subsection{Strukturen}
\subsubsection{Block world}

\begin{center}
  \begin{tikzpicture}[scale=2]
    \node (a) {bb};
    \node [right=of a] (b) {rb};
    \node [above=of b] (c) {gb};
    \node [below=of a] (d) {Block B};
    \node [below=of b] (e) {Block R};
    \node [above=of c] (f) {Block G};
    \draw [<->] (a) -- (b) node[below,pos=0.5] {neben};
    \draw [<-] (c) -- (b) node[left,pos=0.5] {unter};
    \draw [->] (c) to[in=45,out=-45] (b) node[right,pos=0.5] {auf};
    \draw (a) -- (d) (b) -- (e) (c) -- (f);
  \end{tikzpicture}
\end{center}

\subsubsection{Weiteres Beispiel}
$(\mathbb{N}, <, P_1^{\mathfrak{A}}, P_2^{\mathfrak{A}}, \dots)$
\begin{verbatim}
P_1  <          <  P_1,P_2  <         <    P_2
 0 ------> 1 -------> 2 -------> 3 -------> 4 --------> ....
 |         |          |          |
 __________|__________|          |
      <    |                     |
           ______________________|
                     <
\end{verbatim}

\subsubsection{Beispiel Datenbank}
Betrachte eine Datenbank mit 2 Tabellen:
\begin{itemize}
\item Tabelle Film, 3 Spalten
\begin{itemize}
\item Name, Typ String
\item Jahr, Typ pos. Integer
\item Regisseur, Typ String
\end{itemize}
\item Tabelle Schauspieler, 2 Spalten
\begin{itemize}
\item Name, Typ String
\item Filmname, Typ String
\end{itemize}
\end{itemize}

Beispielinstanz: \\ \\
\begin{tabular}{|c|c|c|}
Name & Jahr & Regisseur \\
\hline
Die V"ogel & 1963 & Hitchcock \\
Foosel & 1963 & Hitchcock \\
Goldfinger & 1964 & Hamilton \\
\hline
\end{tabular} \begin{tabular}{|c|c|}
Name & Film \\
\hline
Conney & Foosel \\
Conney & Goldfinger \\
Hedren & Foosel \\
\hline
\end{tabular}

Als Struktur $\mathfrak{A}$:
\begin{eqnarray*}
A &=& \{Die V"ogel, Foosel, Goldfinger, 1963, 1964, Hitchcock, Hamilton, Conney, Hedren\}\\
Film^{\mathfrak{A}} &=& \{ (Die V"ogel, 1963, Hitchcock),
               (Foosel, 1963, Hitchcock),
               (Goldfinger, 1964, Hamilton) \} \\
Schauspieler^{\mathfrak{A}} &=& \{ (Conney, Foosel), (Conney,Goldfinger), (Hedren,Foosel)\}
\end{eqnarray*}

\subsubsection{Beispiel XML}
Als Baum/bin"are Relation:
\begin{eqnarray*}
A &=& \{\epsilon , 0, 1, 00, 01, 11, 000, 001, 010, 011, 110, 111\} \\
succ^{\mathfrak{A}} &=& \{(w, w_i) | w \in A^{*}, i \in \{0,1\}, w_i \in A \} \\
<^{\mathfrak{A}} &=& \{ (w_0, w_1) | w_0,w_1 \in \mathfrak{A}\} \\
Inventory^{\mathfrak{A}} &=& \{ \epsilon \} \\
Price^{\mathfrak{A}} &=& \{ 000, 01,0, 110 \}
\end{eqnarray*}

\subsection{Pr"adikatenlogik Semantik - Therme}
Betrachte folgende Struktur $\mathfrak{A}$ mit un"aren Funktionssymbolen f,g und Konstante c:

\begin{center}
  \begin{tikzpicture}[scale=2]
    \node (a) {$a_1$};
    \node [xshift=4em,below=of a] (b) {$a_2$};
    \node [xshift=-4em,below=of a] (c) {$a_4$};
    \node [xshift=-4em,below=of b] (d) {$a_3$};
    % TODO: kante zu sich selbst
    \draw [->] (a) -- (a) node[above] {f};
    \draw [->] (c) -- (c) node[left] {g};

    \draw [->] (a) -- (b) node[right,pos=0.5] {g};
    \draw [->] (b) -- (d) node[below,pos=0.5] {f};
    \draw [->] (d) -- (c) node[below,pos=0.5] {f,g};
    \draw [->] (c) -- (a) node[left,pos=0.5] {f};
    \draw [->] (b) -- (c) node[below,pos=0.5] {g};
  \end{tikzpicture}
\end{center}

Betrachte:
\begin{itemize}
\item $\beta(x) = a1$, dann $\beta( g( f(x) ) ) = a2$
\item $\beta(x) = a3$, dann $\beta( g( f(x) ) ) = a4$
\end{itemize}


\section{VL vom 23.~November 2010}

\subsection{Beispiele zur Semantik von FO}

\begin{itemize}
  \item Beispiel 1:\par
  $\Afrak$ (Blocksworld-Graph hier einfügen)
  
  Zuweisung $\beta$: $\beta(x) = bb \qquad \beta(y) = gb \qquad \beta(z) = gb$
  % Absicht, da Beispielbelegung: \beta(y) = \beta(z)
  
  \begin{align}
    \Afrak, \beta &\models \exists x.(R(x) \AND \forall y.(\text{auf}(y,x) \IMPL G(y))) && \text{Satz} \\
    \Afrak, \beta &\models \exists z.(\text{neben}(b_1,z) \AND \text{unter}(z,b_2))     && \text{Satz} \\
    \Afrak, \beta &\models \exists z.(\text{neben}(x,z) \AND \text{unter}(z,y))         && \text{kein Satz} \\
    \Afrak, \beta &\models x \not= b_2                                                  && \text{kein Satz} \\
    \Afrak, \beta &\models \forall x.(G(x) \OR B(x) \OR x = \text{lieblingsblock})      && \text{Satz}
  \end{align}
  
  \item Beispiel 2:\par
  $\mathfrak{N} = (\mathbb{N}, +, \cdot, 0, 1)$
  
  Sei $\beta$ beliebig. Dann gelten folgende Sätze:
  \begin{align}
    \mathfrak{N}, \beta &\models \forall x \forall x.(x+y=y+x) \\
                        & \qquad\text{(zur Vereinfachung: $a+b = +(a,b)$)} \notag\\
    \mathfrak{N}, \beta &\models \forall x.(\text{Prim}(x) \IMPL \exists y.(y>x \AND \text{Prim}(y))) \\
                        & \qquad\text{wobei $\text{Prim}(x) = \forall y,z.(x=y\cdot z \IMPL (y=1 \OR z=1))$} \notag\\
                        & \qquad\text{und $y>x = \exists z.(z\not=0 \AND x+z = y)$} \notag\\
    \mathfrak{N}, \beta &\stackrel{?}{\models} \forall x.(\text{Prim}(y) \AND \text{Prim}(y+z)) \\
                        & \qquad\text{wobei $y+z = (y+1)+1$} \notag\\
                        & \qquad\text{\enquote{Gibt es unendlich viele Primzahlzwillinge?}}\notag\\
                        & \qquad\text{(offenes Problem der Zahlentheorie)}\notag
  \end{align}
\end{itemize}

\subsection{Isomorphielemma}

\subsubsection{Beispiel}

$c_1, c_2  \in F^0(\tau) \qquad f \in F^1(\tau) \qquad R \in R^2(\tau)$

\begin{verbatim}
  Graph von Bernd tikzen
\end{verbatim}

\subsubsection{Beweis}

Wegen des Koinzidenzlemmas genügt es, zu zeigen:

Wenn $\mathcal{I}_\Afrak = (\Afrak,\beta)$ und $\mathcal{I}_\mathfrak{B} = (\mathfrak{B}, \beta)$ Interpretationen,
so dass $\beta'(x) = \pi(\beta(x))$ für alle $x\in VAR$, dann

\begin{itemize}
  \item[$(*)$] $\mathcal{I}_\Afrak \models \varphi$ für alle $\varphi \in FO(\tau)$
\end{itemize}

Man zeigt leicht per Induktion über die Struktur über die Struktur von t:

\begin{itemize}
  \item[$(**)$] $\beta'(t) = \pi(\beta(t))$ für alle $t \in T(\tau)$
\end{itemize}

Beweis von $(*)$ per Induktion über die Struktur von $\varphi$

\begin{itemize}
  \item \textbf{Induktionsanfang}
  \begin{enumerate}
    \item $\mathcal{I}_\Afrak \models t_1=t_2$\\
      gdw. $\beta(t_1) = \beta(t_2)$ (Semantik \enquote{=})\\
      gdw. $\pi(\beta(t_1)) = \pi(\beta(t_2))$ ($\pi$ injektiv)\\
      gdw. $\beta'(t_1) = \beta'(t_2)$ ($(**)$)\\
      gdw. $\mathcal{I}_\mathfrak{B} \models t_1=t_2$
    
    \item $\mathcal{I}_\Afrak \models P(t_1,\dots,t_k)$ \\
      gdw. $(\beta(t_1),\dots,\beta(t_k)) \in P^\Afrak$ \\
      gdw. $(\pi(\beta(t_1)),\dots,\pi(\beta(t_k))) \in P^\mathfrak{B}$ ($\pi$ isom.) \\
      gdw. $(\beta'(t_1),\dots,\beta'(t_k)) \in P^\mathfrak{B}$ \\
      gdw. $\mathcal{I}_\mathfrak{B} \models P(t_1,\dots,t_2)$
  \end{enumerate}
  
  \item \textbf{Induktionsschritt}
  \begin{enumerate}
    \item Fälle $\varphi = \NOT \psi$, $\varphi = \psi_1 \AND \psi_2$, $\varphi = \psi_1 \OR \psi_2$
      (Übung)
      
    \item $\mathcal{I}_\Afrak \models \exists x.\psi$ \\
      gdw. $(\Afrak, \beta[x/a]) \models \psi$ für ein $a \in A$\\
      gdw. $(B, \beta'[x/\pi(a)]) \models \psi$ für ein $a \in A$ (IV)\\
      gdw. $(B, \beta'[x/b]) \models \psi$ für ein $b \in B$ ($\pi$ surj.)\\
      gdw. $\mathcal{I}_B \models \exists x.\psi$
      
    \item Fall $\varphi = \forall x.\psi$ analog.
  \end{enumerate}
\end{itemize}
\qed



\section{VL vom 25.~November 2010}



\section{VL vom 30.~November 2010}

\subsection{Beispiel PNF}

\begin{align}
  \varphi &= \NOT\forall x.(R(x,x) \AND \forall x.\exists y.R.(x,y)) && \text{(Fall 3)} \\
    &\EQUIV \NOT\forall x.(R(x,x) \AND \forall z.\exists y.R.(z,y)) && \text{(Variablenumbenennung)} \\
    &\EQUIV \NOT\forall x.\forall z.\exists y.(R(x,x) \AND R(z,y)) && \text{(Negation reinziehen)} \\
    &\EQUIV \exists x.\exists z.\forall y. \NOT(R(x,x) \AND R(z,y))
\end{align}

\subsection{Postsches Korrespondenzproblem (PCP)}

\[
  F = \set{(0,1)_1, (1,10)_2, (01,1)_3}
\]

Indexfolge 2,3 ist Lösung:

\begin{itemize}
  \item linke Konkatenation:  $1|0 1$
  \item rechte Konkatenation: $1 0|1$
\end{itemize}

\subsubsection{Unentscheidbarkeit}

\begin{verbatim}
  Baum von Carsten TikZen
\end{verbatim}

\subsubsection{Beweis des Lemmas Unentscheidbarkeit}

\begin{description}
  \item[\enquote{$\Leftarrow$}]
  Sei $\varphi_F$ fültig. Betrachte folgende Struktur:
  \begin{align*}
    \Afrak &= (A, c_\epsilon, f_0, f_1, P) \qquad\text{mit} \\
    A &= \set{0,1}^*\\
    c_\epsilon^{\Afrak} &= \epsilon\\
    f_0^{\Afrak}(w) &= w \cdot 0\\
    f_1^{\Afrak}(w) &= w \cdot 1\\
    P &= \set{(u,v) \mid \text{es gibt $i_1,\dots,i_l$ mit $u=u_{i_1}\cdots u_{i_l}$ und $v=v_{i_1}\cdots v_{i_l}$}}
  \end{align*}
  
  Es gilt
  \begin{enumerate}
    \item $\Afrak \models \varphi$ (nach Definition von $P$ und wegen $t_w(c_\epsilon)^{\Afrak}=w$ für alle $w\in\set{0,1}^*$)
    \item $\Afrak \models \psi$: wenn $(u,v)\in P^{\Afrak}$, dann folgt mit Definition von $P$ auch $P(uu_i, vv_i)$ für $1\leq i\leq h$.
    
    Wegen $t_w^\Afrak(w') = w'w$ für alle $w,w'\in\set{0,1}^*$, also $P(t_{u_i}^\Afrak(u), t_{v_i}^\Afrak(v))$.

    Wegen Gültigkeit von $\varphi_F$, also $\Afrak \models \exists x.P(x,x)$. Nach Definition von $P$ hat $F$ aleo eine Lösung.
  \end{enumerate}
  
  \item[\enquote{$\Rightarrow$}]
  Sei $i_1,\dots,i_l$ Lösung für $F$ und $\Afrak = (A, c, f_1, f_2, P)$ eine Struktur. Zu zeigen: $\Afrak\models\varphi_F$.
  
  Wenn $\Afrak\not\models \varphi\AND\psi$, gilt $\Afrak\models\varphi_F$. Es gelte $\Afrak\models\varphi\AND\psi$.
  Definiere Abbildung $h : \set{0,1}^* \rightarrow A$:
  \begin{align*}
    h(\epsilon) &= c_\epsilon^\Afrak\\
    h(w0) &= f_0^{\Afrak}(h(w))\\
    h(w1) &= f_1^{\Afrak}(h(w))
  \end{align*}
  
  Man sieht leicht, dass $h(w)=t_w(c_\epsilon)^\Afrak$ für alle $w\in\set{0,1}^*$.
  
  Wegen $\Afrak \models \varphi$, also $(h(u_{i_1}), h(v_{i_1}) \in P^{\Afrak}$.
  
  Wegen $\Afrak \models \psi$ können wir unduktiv schließen, dass $(h(u_{i_1}\cdots u_{i_r}), h(v_{i_1}\cdots v_{i_r}))\in P^{\Afrak}$ für $1\leq r\leq l$.
  
  Sei $u_{i_1}\cdots u_{i_r} = v_{i_1}\cdots v_{i_r} = w$. Es gilt also $(h(w),h(w)) \in P^{\Afrak}$, damit $\Afrak \models \exists x.P(x,x)$, also $\Afrak\models\varphi_F$.
\end{description}

\subsection{Beweis Theorem}

Wir nehmen o.B.d.A. an, dass $\varphi$ ein Satz ist (universeller Abschluss).

\begin{description}
  \item[\enquote{$\Leftarrow$}]
  Angenommen, $\varphi$ ist nicht gültig. Dann gibt es Struktur $\Afrak$ mit $\Afrak\not\models\varphi$.
  Erweitere $\Afrak$ um $P^\Afrak = \set{(a,a)\mid a\in A}$.
  Offensichtlich gilt $\Afrak\models\ANDop\Gamma_\tau$. Man zeigt leicht, dass $\Afrak,\beta\models\psi$ gdw. $\Afrak,\beta\models\psi[P_=/=]$ für alle $\psi\in FO(\tau)$ und Zuweisungen $\beta$. Wegen $\Afrak\not\models\varphi$, also $\Afrak\not\models\varphi[P_=/=]$, also $\Afrak\not\models\ANDop\Gamma_\tau \rightarrow \varphi[P_=/=]$, damit ist diese Formel nicht gültig.
  
  \item[\enquote{$\Rightarrow$}]
  Angenommen, $\ANDop\Gamma_\tau \rightarrow \varphi[P_=/=]$ ist nicht gültig. Dann gibt es Struktur $\Afrak$ mit $\Afrak\not\models\ANDop\Gamma_\tau\rightarrow\varphi[P_=/=]$. Dann $\Afrak\models\ANDop\Gamma_\tau$ und $\Afrak\not\models\varphi[P_=/=]$. Wegen ersterem ist $P_=^\Afrak$ Äquivalenzrelation auf $A$. Für $a\in A$ sei $[a]$ die Äquivalenzklasse von $a$. Definiere Struktur $\hat{\Afrak}$ (Quotientenstruktur):
  \begin{align*}
    \hat{A} &= \set{[a] \mid a\in A}\\
    R^{\hat{\Afrak}} &= \set{([a_1],\dots,[a_n]) \mid (a_1,\dots,a_n) \in R^\Afrak }\\
  \end{align*}
  
  Für jede Zuweisung $\beta$ in $\Afrak$ sei $\hat{\beta}$ die Zuweisungen in $\hat{\Afrak}$ mit $\hat{\beta}(x)=[\beta(x)]$. Man zeigt per Induktion über die Struktur von $\psi$
  \begin{itemize}
    \item[$(*)$] Für alle $\psi\in FO(\tau)$ und alle Zuweisungen $\beta$ gilt $\Afrak,\beta\models\psi[P_=/=]$ gdw. $\hat{\Afrak},\hat{\beta}\models\psi$.
  \end{itemize}
  Wegen $\Afrak\not\models\varphi[P_=/=]$ folgt $\hat{\Afrak}\not\models\varphi$, also ist $\varphi$ nicht gültig.
\end{description}
\qed

\section{VL vom 07. Dezember 2010}

\subsection{FO Theorien}

\begin{enumerate}
  \item $\Taut(\tau)$ ist Theorie:
  \begin{enumerate}
    \item erfüllbar, denn $\mathfrak{A}\models\Taut(\tau)$ für jede Struktur $\mathfrak{A}$
    \item Wenn $\Taut(\tau)\models\psi$ erfüllbar, dann ist $\varphi$ Tautologie, also $\varphi\in\Taut(\tau)$
  \end{enumerate}
  
  Nicht vollständig, denn es gibt Sätze $\varphi$, die weder Tautologie sind noch unerfüllbar
  (also $\NOT\varphi$ keine Tautologie).
  
  \item $\cl(\emptyset)$ ist genau $\Taut(\tau)$, also nicht vollständig.
\end{enumerate}



\section{VL von 09.~Dezember 2010}

\subsection{Sequenzkalkül: Seitenbedingung}

\textbf{Seitenbedingung ($\exists\IMPL$)}:

Diese Bedingung ist notwendig, weil die Regel sonst nicht korrekt wäre.

Beispiel:

\[
  \frac{
    \psi[c], \varphi[c] \IMPL \psi\AND\varphi[c] \qquad\text{(gültig)}
  }{
    \psi[c], \exists x.\varphi(x) \IMPL \psi\AND\varphi[c] \qquad\text{(nicht gültig)}
  }
\]

Für ($\IMPL\exists$) braucht man keine Seitenbedingung, z.B.

\[
  \frac{
    \psi[c], \varphi[c] \IMPL \psi\AND\varphi[c]  \qquad\text{(gültig)}
  }{
    \psi[c], \varphi[c] \IMPL \exists x.(\psi\AND\varphi(x)) \qquad\text{(gültig)}
  }
\]

\subsection{Sequenzkalkül: Beispiele}

Beispiele für ableitbare Sequenzen:

\begin{align*}
  \underline{P(c)}, Q(c) &\IMPL \underline{P(c)}, R(c) && \text{\underline{Axiom}} \\
  P(c), Q(c) &\IMPL P(c), Q(c) && \text{Axiom}\\
  P(c), Q(c) &\IMPL P(c), Q(c), R(c) && \text{$\IMPL\AND$-Regel}
\end{align*}

\subsection{SK-Beweise}

\begin{verbatim}
  Ableitungsbäume von Carsten tikzen (?)
\end{verbatim}

\subsection{Korrektheitsbeweis}

Zwei Regeln exemplarisch:

\begin{description}
  \item[$\OR\IMPL$]
  Seien $\Gamma,\psi\IMPL\Delta$ und $\Gamma,\psi\IMPL\Delta$ gültig.
  Zu zeigen: $\Gamma,\varphi\OR\psi\IMPL\Delta$ gültig. Es gelte $\Afrak\models\Gamma,\varphi\OR\psi$.
  Dann per Semantikk auch $\Afrak\models\psi$ oder $\Afrak\models\Delta$ wegen Gültigkeit der
  ursprünglichen Sequenzen.
  
  \item[$\exists\IMPL$]
  Sei $\Gamma,\varphi[c]\IMPL\Delta$ gültig. Zu zeigen: $\Gamma,\exists x.\varphi(x)\IMPL\Delta$ gültig.
  Es gelte $\Afrak\models\Gamma,\exists x.\varphi(x)$. Dann gibt es $a\in A$ mit $\Afrak,\set{x\mapsto a}\models\varphi(a)$.
  Sei $\Afrak'$ wie $\Afrak$, aber mit $c^{\Afrak'}=a$. Da $c$ nicht in $\Gamma$ und $\varphi(x)$ vorkommt,
  gilt $\Afrak'\models\Gamma,\varphi[c]$, also $\Afrak'\models\Delta$. Da $c$ nicht in $\Delta$ vorkommt,
  folgt $\Afrak\models\Gamma$.
\end{description}



\section{VL von 14.~Dezember 2010}

\subsection{Beispiel Herbrand-Struktur}

Sei $\tau = \set{c, f_1, f_2, P}$, wobei $c\in F^0(\tau)$ (Konstantensymbol)
$f_1,f_2 \in F^1(\tau)$ und $P \in R^2(\tau)$.

Für alle Herbrand-Strukturen für $\tau$ gilt:

\begin{align*}
  H &= \set{c, f_1(c), f_2(c), f_1(f_1(c)), f_1(f_2(c)), \dots} \\
  c^\Hfrak &= c \qquad f_1^\Hfrak(c) = f_1(c) \qquad f_2^\Hfrak(c) = f_2(c) \qquad \dots
\end{align*}

Die Interpretation von P ist frei, z.B.

\begin{verbatim}
  Baum-/Graph-Struktur von Bernd/Carsten tikzen
\end{verbatim}

Es genügt, zu zeigen

\begin{itemize}
  \item[$(*)$] Für alle Sequenzen $\Gamma\IMPL\Delta$ und deren reduzierte Varianten
  $\Gamma'\IMPL\Delta'$ gilt: Jeder SK-Beweis für $\Gamma'\IMPL\Delta'$ kann in SK-Beweis
  für $\Gamma\IMPL\Delta$ gewandelt werden.
  
  Denn dann gilt:
  
  \begin{align*}
                            &\models\Gamma \IMPL\Delta \\
    \text{impliziert} \quad &\models\Gamma'\IMPL\Delta' && \text{(Äquivalenz)} \\
    \text{impliziert} \quad &\vdash \Gamma'\IMPL\Delta' && \text{(alle gültigen reduzierten Sequenzen ableitbar)} \\
    \text{impliziert} \quad &\vdash \Gamma \IMPL\Delta  && (*)
  \end{align*}
\end{itemize}

Beweisidee $(*)$:

Um SK-Beweis für $\Gamma\IMPL\Delta$ zu erhalten, folge derselben Strategie wie im
Beweis von $\Gamma'\IMPL\Delta'$. Jede auftretende Teilformel, die nicht reduziert
ist, wird mittels SK-Beweisen für

\begin{align*}
  \varphi \OR \psi &\EQUIV \NOT(\NOT\varphi \AND \NOT\psi) \\
  \forall x.\varphi(x) &\EQUIV \NOT\exists x.\NOT\varphi(x)
\end{align*}

\enquote{on the fly} in reduzierte Form gewandelt.

\subsection{??}

Wir konstruieren zwei Folgen:

\Gamma_0 \subseteq \Gamma_1 \subseteq \dots und
\Delta_0 \subseteq \Delta_1 \subseteq \dots und setzen

\Gamma* = \bigcup_{i\geq 0} \Gamma_i und
\Delta* = \bigcup_{i\geq 0} \Delta_i

Diese Konstruktion stellt sicher, dass
\begin{itemize}
  \item[(*)] \Gamma_n \IMPL \Delta_n nicht ableitbar für alle n\geq 0.
\end{itemize}

Wir beginnen mit \Gamma_0=\Gamma und \Delta_0=\Delta.

Da \tau abzählbar, gibt es Aufzählung
  (\varphi_0, t_0), (\varphi_1, t_1), \dots,
so dass jedes Paar (\varphi,t) mit \varphi Satz aus FO(\tau) und t Grundterm
aus T(\tau) unendlich oft vorkommt.

\begin{verbatim}
  "Dovetailing" + Reset
  tikzen
  
  (\varphi_0, t_0),
  (\varphi_0, t_0),(\varphi_1, t_1),
  (\varphi_0, t_0),(\varphi_1, t_1),(\varphi_2, t_2),
  (\varphi_0, t_0),(\varphi_1, t_1),(\varphi_2, t_2),(\varphi_3, t_3),
  :
  .
\end{verbatim}

Für $n\geq 0$ konstruieren $\Gamma_{n+1}, \Delta_{n+1}$ wie folgt:

\begin{enumerate}
  \item $\varphi_n \in \NOT\psi$\\
  Wenn $\varphi_n \in \Gamma_n$, dann $\Gamma_{n+1}=\Gamma_n, \Delta_{n+1}=\Delta_n\cup\set{\psi}$.\\
  Wenn $\varphi_n \in \Delta_n$, dann $\Gamma_{n+1}=\Gamma_n\cup\set{\psi}, \Delta_{n+1}=\Delta_n$.\\
  (Sonst: $\Gamma_{n+1}=\Gamma_n, \Delta_{n+1}=\Delta_n$)
  
  Mit den Regeln ($\NOT\IMPL$) und ($\IMPL\NOT$) folgt, dass $\Gamma_{n+1}\IMPL\Delta_{n+1}$ nicht ableitbar.
  Exemplarisch 1. Fall:
  
  Angenommen, $\Gamma_n\IMPL\Delta_n\cup\set{\psi}$ ist ableitbar. Dann mit ($\NOT\IMPL$) auch $\Gamma_n\cup\set{\NOT\psi}\IMPL\Delta_n = \Gamma_n\IMPL\Delta_n$. $\lightning$
  
  \item $\varphi_n = \psi_1 \AND \psi_2$\\
  Wenn $\varphi_n \in \Gamma_n$, dann $\Gamma_{n+1}=\Gamma_n\cup\set{\psi_1, \psi_2}, \Delta_{n+1}=\Delta_n$.
  Dann $\Gamma_{n+1}\IMPL\Delta_{n+1}$ nicht ableitbar wegen ($\AND\IMPL$).
  
  Wenn $\varphi_n\in\Delta_n$, dann ist
    (i) $\Gamma_n \in \Delta_n\cup\set{\psi_1}$ oder
    (ii) $\Gamma_n\IMPL\Delta_n\cup\set{psi_2}$ nicht ableitbar,
  denn sonst wäre ($\IMPL\AND$) auch $\Gamma_n\IMPL\Delta_n$ ableitbar.
  
  Setze \Gamma_{n+1}=\Gamma_n und \Delta_{n+1}=\Delta_n\cup\set{\psi_1}.
  Wenn Fall (i) zutrifft.
  
  \item \varphi_n = \exists x.\varphi(x)\\
  Wenn \varphi_n\in\Gamma_n, dann wähle c\in C, welches nicht in \Gamma_n\cup\Delta_n
  vorkommt. Setze \Gamma_{n+1}=\Gamma_n\cup\set{\psi(c)}, \Delta_{n+1}=\Delta_n.
  
  Wenn \varphi_n\in\Delta_n, dann setze \Gamma_{n+1}=\Gamma_n und \Delta_{n+1}=\Delta_n\cup\set{\psi(t_n)}.
\end{enumerate}

Mit den Regeln (\exists\IMPL) und (\IMPL\exists) folgt, dass \Gamma_{n+1}\IMPL\Delta_{n+1} nicht ableitbar.

Es bleibt, zu überprüfen, dass \Gamma^*,\Delta^* die Bedingungen 1--4 des Lemmas erfüllen. Exemplarisch 1+4:

\begin{enumerate}
  \item Wenn \varphi\in\Gamma^*\cap\Delta^*, dann gibt es n\geq 0 mit \varphi\in\Gamma_n\cap\Delta_n. Dann ist \Gamma_n\IMPL\Delta_n ein Axiom und damit ableitbar, im Widerspruch zu (*).
  
  \item[4.] Wenn \exists x.\varphi(x)\in\Gamma^*, dann \exists x.\psi(x)\in\Gamma_m für ein m\geq 0. Dann gibt es ein n\geq m mit \varphi_n=\exists x.\psi(x). Nach (c) gilt \psi(c)\in\Gamma_{n+1}\subseteq\Gamma^* für eine Konstante c.
  
  Wenn \exists x.\psi(x)\in\Delta^* und t beliebiger Grundterm, dann gibt es n\geq 0 mit \exists x.\psi(x)\in\Delta_n, \varphi_n=\exists x.\psi(x) und t_n=t. Also \psi(t_n)\in\Delta_{n+1}\in\subseteq\Delta^*.
\end{enumerate}
\qed

\subsubsection{Vollständig (kontrapositiv)}

Sei \Gamma\IMPL\Delta nicht ableitbar, \Gamma^* und \Delta^* wie im Lemma.

\section{VL von 6.~Januar 2011}




\end{document}

