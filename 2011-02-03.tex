\section{VL von 03.~Februar 2011}

\subsection{Beispiel}

\[
  \varphi = \exists X,Y.(X\subseteq P_1 \AND Y\subseteq P_2 \AND \suc(X) = Y)
\]

Die Sprache $\mathcal{L}(\varphi)\subseteq\Sigma_2=\set{0,1}^2$ ist
\[
  \Sigma_2^* \cdot \left(
    \begin{pmatrix} 1\\0 \end{pmatrix}
    \cup
    \begin{pmatrix} 1\\1 \end{pmatrix}
  \right)
  \cdot \left(
    \begin{pmatrix} 0\\1 \end{pmatrix}
    \cup
    \begin{pmatrix} 1\\1 \end{pmatrix}
  \right)
  \cdot \Sigma_2^*
\]

Zur Erinnerung: jedes Relationssymbol $P_i$ und jede freie Variable
wird durch ein \enquote{Bit} repräsentiert.

Reihenfolge: $X,Y,P_1,P_2$

\begin{itemize}
  \item $A_{X\subseteq P_1}$
  \begin{verbatim}
    Automat tikzen
  \end{verbatim}

  \item $A_{Y\subseteq P_2}$
  \begin{verbatim}
    Automat tikzen
  \end{verbatim}

  \item $A_{\suc(x)=Y}$
  \begin{verbatim}
    Automat tikzen
  \end{verbatim}

  \item $X\subseteq P_1 \AND Y\subseteq P_2 \AND \suc(X)=Y$
  \begin{enumerate}
    \item Erweitern der Signaturen/Spuren (Reihenfolge siehe oben)
    \item Produktautomat bauen:
    \begin{verbatim}
      Automat tikzen
    \end{verbatim}
  \end{enumerate}
  
  \item $A_\varphi$: Projektion
  \begin{verbatim}
    Automat tikzen
  \end{verbatim}  
\end{itemize}

\subsection{Beispiele sternfreier Sprachen}

\begin{align*}
  \Sigma^* 
    \quad\text{dargestellt}\quad&
    \overline{\emptyset} \\
  (10)^* 
    \quad\text{dargestellt}\quad&
    (\set{1}\circ\Sigma^*\circ\set{0}) \cap
      \overline{(\Sigma^*\circ\set{0}\circ\set{0}\circ\Sigma^*)} \cap
      \overline{(\Sigma^*\circ\set{1}\circ\set{1}\circ\Sigma^*)} \\
    \EQUIV\quad&
      (1\Sigma^*0) \cap \overline{(\Sigma^*00\Sigma^*)} \cap \overline{(\Sigma^*11\Sigma^*)} \\
  1^*0^*\cup 0^*1^*
    \quad\text{dargestellt}\quad&
    \overline{\Sigma^*01\Sigma^*} \cup \overline{\Sigma^*10\Sigma^*}
\end{align*}

\subsection{LTL -- Beispiele}

\[
  \bigcirc\varphi \qquad \diamondsuit\varphi \qquad \square\varphi \qquad \Until{\varphi}{\psi}
\]

Der Wartezustand wird nur durch Ausführen des kritischen Abschnittes beendet:

\[
  \ANDop_{i\in\set{1,2}} \square (W_i \IMPL (\Until{W_i}{A_i} \OR \square W_i))
\]

Prozesse warten nicht grundlos:

\begin{align*}
                    & \square ((W_1\OR W_2) \IMPL (A_1\OR A_2)) \\
  \text{oder} \quad & \square ((W_1\OR W_2) \IMPL \bigcirc(A_1\OR A_2))
\end{align*}

\subsubsection{Semantik LTL}

\begin{verbatim}
  |-------+-------+-------+-------+-------|
          |       p
     \bigcirc p

  |-------+-------+-------+-------+-------|
          |               p
    \diamondsuit p

  |-------+-------+-------+-------+-------|
          |       p       p       p       p
      \square p

  |-------+-------+-------+-------+-------|
          |       p       p       q
        p U q
          p
\end{verbatim}

Es gelten die Äquivalenten
\begin{itemize}
  \item $\diamondsuit \varphi \EQUIV \Until{true}{\varphi}$ mit $true = p\AND\NOT p$,
  \item $\square \varphi \EQUIV \NOT\diamondsuit\NOT \varphi$.
\end{itemize}

Es genügt also eigentlich auch $\bigcirc$ und $\mathcal{U}$ als temporale Operatoten.


\subsubsection{F1S}

Wir übersetzen LTL-Formeln $\varphi$ in F1S-Formel $\widehat{\varphi}(x)$ mit freier
Variable $x$, per Induktion über die Struktur von $\varphi$:

\begin{align*}
  \widehat{p_i}(x) &= P_i(x) \\
  \widehat{\NOT\varphi}(x) &= \NOT\widehat{\varphi}(x) \\
  \widehat{\varphi\AND\psi}(x) &= \widehat{\varphi}(x)\AND\widehat{\psi}(x) && \text{ebenso für $\OR$} \\
  \widehat{\bigcirc\varphi}(x) &= \exists y.(y=s(x) \AND \widehat{\varphi}(y)) \\
  \widehat{\diamondsuit\varphi}(x) &= \exists y.(x\subseteq y \AND \widehat{\varphi}(y)) \\
  \widehat{\square\varphi}(x) &= \forall y.(x\subseteq y \IMPL \widehat{\varphi}(y)) \\
  \widehat{\Until{\varphi}{\psi}}(x) &= \exists y.(x\subseteq y \AND \widehat{\psi}(y) \AND \forall z.(x\subseteq z < y && \text{\textcolor{red}{unvollständig!}}
\end{align*}

Die konstruierte Formel ist sogar vollständig äquivalent, als
$\Afrak,n\models\varphi$ gdw. $\Afrak\models\hat{\varphi}[n]$ für alle
$\Afrak$ \textbf{und alle $\mathbf{n}$}.
