\section{VL vom 26.~Oktober 2010}

Wir verwenden die Variablen 

\begin{align}
  x_{ij}^\alpha \quad\text{mit}\quad \alpha \in \set{M,S,K}, i \in \set{\I,\II,\III}, j \in \set{a,b}
\end{align}

Zum Beispiel repräsentiert die Variable $x_{\II a}^M$ die Aussage
\enquote{Müller unterrichtet in Stunde \II Klasse $a$}.

Wir modellieren nun das Problem wie folgt:

\begin{itemize}
  \item Jede Stunde wurd von einem passendem Lehrer unterrichtet
  \begin{align}
    \varphi_1 &=
      \ANDop_{ij \in\set{\I a,\II a,\III b}} \biggl(x_{ij}^M \OR x_{ij}^K\biggr) \AND
      \ANDop_{ij \in\set{\I b,\II a,\II b}}  \biggl(x_{ij}^S \OR x_{ij}^K\biggr)
  \end{align}
  
  \item Jede Stunde wird von höchstens einem Lehrer unterrichtet:
  \begin{align}
    \varphi_2 &=
      \ANDop_{i\in\set{\I,\II,\III}} \ANDop_{j\in\set{a,b}}
        \NOT \biggl(x_{ij}^M \AND x_{ij}^S\biggr) \AND
        \NOT \biggl(x_{ij}^M \OR x_{ij}^K\biggr) \AND
        \NOT \biggl(x_{ij}^S \OR x_{ij}^K\biggr)
  \end{align}
  
  \item Jeder Lehrer unterrichtet mindesten zwei Stunden:
  \begin{align}
    \varphi_3 &=
      \ANDop_{\alpha\in\set{M,K,S}} \ORop_{\substack{i,i'\in\set{\I,\II,\III}\\ j,j'\in\set{a,b}\\ ij\not=i'j'}}
        \biggl(x_{ij}^\alpha \AND x_{i'j'}^\alpha\biggr)
  \end{align}
  
  \item Ein Lehrer kann nur eine Klasse zur Zeit unterrichten
  \begin{align}
    \varphi_4 &=
      \ANDop_{\alpha\in\set{M,K,S}} \ANDop_{i\in\set{\I,\II,\III}}
        \NOT \biggl(x_{ia}^\alpha \AND x_{ib}^\alpha\biggr)
  \end{align}
\end{itemize}

Man sieht nun leicht, dass die möglichen Lösungen für das Zeitplanungsproblem
ganau dem Belegungen $V$ entsprechen, die
$\varphi_1 \AND \varphi_2 \AND \varphi_3 \AND \varphi_4$ erfüllen.
