\section{VL von 09.~Dezember 2010}

\subsection{Sequenzkalkül: Seitenbedingung}

\textbf{Seitenbedingung ($\exists\IMPL$)}:

Diese Bedingung ist notwendig, weil die Regel sonst nicht korrekt wäre.

Beispiel:

\[
  \frac{
    \psi[c], \varphi[c] \IMPL \psi\AND\varphi[c] \qquad\text{(gültig)}
  }{
    \psi[c], \exists x.\varphi(x) \IMPL \psi\AND\varphi[c] \qquad\text{(nicht gültig)}
  }
\]

Für ($\IMPL\exists$) braucht man keine Seitenbedingung, z.B.

\[
  \frac{
    \psi[c], \varphi[c] \IMPL \psi\AND\varphi[c]  \qquad\text{(gültig)}
  }{
    \psi[c], \varphi[c] \IMPL \exists x.(\psi\AND\varphi(x)) \qquad\text{(gültig)}
  }
\]

\subsection{Sequenzkalkül: Beispiele}

Beispiele für ableitbare Sequenzen:

\begin{align*}
  \underline{P(c)}, Q(c) &\IMPL \underline{P(c)}, R(c) && \text{\underline{Axiom}} \\
  P(c), Q(c) &\IMPL P(c), Q(c) && \text{Axiom}\\
  P(c), Q(c) &\IMPL P(c), Q(c), R(c) && \text{$\IMPL\AND$-Regel}
\end{align*}

\subsection{SK-Beweise}

\begin{verbatim}
  Ableitungsbäume von Carsten tikzen (?)
\end{verbatim}

\subsection{Korrektheitsbeweis}

Zwei Regeln exemplarisch:

\begin{description}
  \item[$\OR\IMPL$]
  Seien $\Gamma,\psi\IMPL\Delta$ und $\Gamma,\psi\IMPL\Delta$ gültig.
  Zu zeigen: $\Gamma,\varphi\OR\psi\IMPL\Delta$ gültig. Es gelte $\Afrak\models\Gamma,\varphi\OR\psi$.
  Dann per Semantikk auch $\Afrak\models\psi$ oder $\Afrak\models\Delta$ wegen Gültigkeit der
  ursprünglichen Sequenzen.
  
  \item[$\exists\IMPL$]
  Sei $\Gamma,\varphi[c]\IMPL\Delta$ gültig. Zu zeigen: $\Gamma,\exists x.\varphi(x)\IMPL\Delta$ gültig.
  Es gelte $\Afrak\models\Gamma,\exists x.\varphi(x)$. Dann gibt es $a\in A$ mit $\Afrak,\set{x\mapsto a}\models\varphi(a)$.
  Sei $\Afrak'$ wie $\Afrak$, aber mit $c^{\Afrak'}=a$. Da $c$ nicht in $\Gamma$ und $\varphi(x)$ vorkommt,
  gilt $\Afrak'\models\Gamma,\varphi[c]$, also $\Afrak'\models\Delta$. Da $c$ nicht in $\Delta$ vorkommt,
  folgt $\Afrak\models\Gamma$.
\end{description}


