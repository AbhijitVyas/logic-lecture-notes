\section{VL vom 30.~November 2010}

\subsection{Beispiel PNF}

\begin{align}
  \varphi &= \NOT\forall x.(R(x,x) \AND \forall x.\exists y.R.(x,y)) && \text{(Fall 3)} \\
    &\EQUIV \NOT\forall x.(R(x,x) \AND \forall z.\exists y.R.(z,y)) && \text{(Variablenumbenennung)} \\
    &\EQUIV \NOT\forall x.\forall z.\exists y.(R(x,x) \AND R(z,y)) && \text{(Negation reinziehen)} \\
    &\EQUIV \exists x.\exists z.\forall y. \NOT(R(x,x) \AND R(z,y))
\end{align}

\subsection{Postsches Korrespondenzproblem (PCP)}

\[
  F = \set{(0,1)_1, (1,10)_2, (01,1)_3}
\]

Indexfolge 2,3 ist Lösung:

\begin{itemize}
  \item linke Konkatenation:  $1|0 1$
  \item rechte Konkatenation: $1 0|1$
\end{itemize}

\subsubsection{Unentscheidbarkeit}

\begin{verbatim}
  Baum von Carsten TikZen
\end{verbatim}

\subsubsection{Unentscheidbarkeit II}


