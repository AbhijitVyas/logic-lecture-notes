\section{VL vom 30.~November 2010}

\subsection{Beispiel PNF}

\begin{align}
  \varphi &= \NOT\forall x.(R(x,x) \AND \forall x.\exists y.R.(x,y)) && \text{(Fall 3)} \\
    &\EQUIV \NOT\forall x.(R(x,x) \AND \forall z.\exists y.R.(z,y)) && \text{(Variablenumbenennung)} \\
    &\EQUIV \NOT\forall x.\forall z.\exists y.(R(x,x) \AND R(z,y)) && \text{(Negation reinziehen)} \\
    &\EQUIV \exists x.\exists z.\forall y. \NOT(R(x,x) \AND R(z,y))
\end{align}

\subsection{Postsches Korrespondenzproblem (PCP)}

\[
  F = \set{(0,1)_1, (1,10)_2, (01,1)_3}
\]

Indexfolge 2,3 ist Lösung:

\begin{itemize}
  \item linke Konkatenation:  $1|0 1$
  \item rechte Konkatenation: $1 0|1$
\end{itemize}

\subsubsection{Unentscheidbarkeit}

\begin{verbatim}
  Baum von Carsten TikZen
\end{verbatim}

\subsubsection{Beweis des Lemmas Unentscheidbarkeit}

\begin{description}
  \item[\enquote{$\Leftarrow$}]
  Sei $\varphi_F$ fültig. Betrachte folgende Struktur:
  \begin{align*}
    \mathfrak{A} &= (A, c_\epsilon, f_0, f_1, P) \qquad\text{mit} \\
    A &= \set{0,1}^*\\
    c_\epsilon^{\mathfrak{A}} &= \epsilon\\
    f_0^{\mathfrak{A}}(w) &= w \cdot 0\\
    f_1^{\mathfrak{A}}(w) &= w \cdot 1\\
    P &= \set{(u,v) \mid \text{es gibt $i_1,\dots,i_l$ mit $u=u_{i_1}\cdots u_{i_l}$ und $v=v_{i_1}\cdots v_{i_l}$}}
  \end{align*}
  
  Es gilt
  \begin{enumerate}
    \item $\mathfrak{A} \models \varphi$ (nach Definition von $P$ und wegen $t_w(c_\epsilon)^{\mathfrak{A}}=w$ für alle $w\in\set{0,1}^*$)
    \item $\mathfrak{A} \models \psi$: wenn $(u,v)\in P^{\mathfrak{A}}$, dann folgt mit Definition von $P$ auch $P(uu_i, vv_i)$ für $1\leq i\leq h$.
    
    Wegen $t_w^\mathfrak{A}(w') = w'w$ für alle $w,w'\in\set{0,1}^*$, also $P(t_{u_i}^\mathfrak{A}(u), t_{v_i}^\mathfrak{A}(v))$.

    Wegen Gültigkeit von $\varphi_F$, also $\mathfrak{A} \models \exists x.P(x,x)$. Nach Definition von $P$ hat $F$ aleo eine Lösung.
  \end{enumerate}
  
  \item[\enquote{$\Rightarrow$}]
  Sei $i_1,\dots,i_l$ Lösung für $F$ und $\mathfrak{A} = (A, c, f_1, f_2, P)$ eine Struktur. Zu zeigen: $\mathfrak{A}\models\varphi_F$.
  
  Wenn $\mathfrak{A}\not\models \varphi\AND\psi$, gilt $\mathfrak{A}\models\varphi_F$. Es gelte $\mathfrak{A}\models\varphi\AND\psi$.
  Definiere Abbildung $h : \set{0,1}^* \rightarrow A$:
  \begin{align*}
    h(\epsilon) &= c_\epsilon^\mathfrak{A}\\
    h(w0) &= f_0^{\mathfrak{A}}(h(w))\\
    h(w1) &= f_1^{\mathfrak{A}}(h(w))
  \end{align*}
  
  Man sieht leicht, dass $h(w)=t_w(c_\epsilon)^\mathfrak{A}$ für alle $w\in\set{0,1}^*$.
  
  Wegen $\mathfrak{A} \models \varphi$, also $(h(u_{i_1}), h(v_{i_1}) \in P^{\mathfrak{A}}$.
  
  Wegen $\mathfrak{A} \models \psi$ können wir unduktiv schließen, dass $(h(u_{i_1}\cdots u_{i_r}), h(v_{i_1}\cdots v_{i_r}))\in P^{\mathfrak{A}}$ für $1\leq r\leq l$.
  
  Sei $u_{i_1}\cdots u_{i_r} = v_{i_1}\cdots v_{i_r} = w$. Es gilt also $(h(w),h(w)) \in P^{\mathfrak{A}}$, damit $\mathfrak{A} \models \exists x.P(x,x)$, also $\mathfrak{A}\models\varphi_F$.
\end{description}

\subsection{Beweis Theorem}

Wir nehmen o.B.d.A. an, dass $\varphi$ ein Satz ist (universeller Abschluss).

\begin{description}
  \item[\enquote{$\Leftarrow$}]
  Angenommen, $\varphi$ ist nicht gültig. Dann gibt es Struktur $\mathfrak{A}$ mit $\mathfrak{A}\not\models\varphi$.
  Erweitere $\mathfrak{A}$ um $P^\mathfrak{A} = \set{(a,a)\mid a\in A}$.
  Offensichtlich gilt $\mathfrak{A}\models\ANDop\Gamma_\tau$. Man zeigt leicht, dass $\mathfrak{A},\beta\models\psi$ gdw. $\mathfrak{A},\beta\models\psi[P_=/=]$ für alle $\psi\in FO(\tau)$ und Zuweisungen $\beta$. Wegen $\mathfrak{A}\not\models\varphi$, also $\mathfrak{A}\not\models\varphi[P_=/=]$, also $\mathfrak{A}\not\models\ANDop\Gamma_\tau \rightarrow \varphi[P_=/=]$, damit ist diese Formel nicht gültig.
  
  \item[\enquote{$\Rightarrow$}]
  Angenommen, $\ANDop\Gamma_\tau \rightarrow \varphi[P_=/=]$ ist nicht gültig. Dann gibt es Struktur $\mathfrak{A}$ mit $\mathfrak{A}\not\models\ANDop\Gamma_\tau\rightarrow\varphi[P_=/=]$. Dann $\mathfrak{A}\models\ANDop\Gamma_\tau$ und $\mathfrak{A}\not\models\varphi[P_=/=]$. Wegen ersterem ist $P_=^\mathfrak{A}$ Äquivalenzrelation auf $A$. Für $a\in A$ sei $[a]$ die Äquivalenzklasse von $a$. Definiere Struktur $\hat{\mathfrak{A}}$ (Quotientenstruktur):
  \begin{align*}
    \hat{A} &= \set{[a] \mid a\in A}\\
    R^{\hat{\mathfrak{A}}} &= \set{([a_1],\dots,[a_n]) \mid (a_1,\dots,a_n) \in R^\mathfrak{A} }\\
  \end{align*}
  
  Für jede Zuweisung $\beta$ in $\mathfrak{A}$ sei $\hat{\beta}$ die Zuweisungen in $\hat{\mathfrak{A}}$ mit $\hat{\beta}(x)=[\beta(x)]$. Man zeigt per Induktion über die Struktur von $\psi$
  \begin{itemize}
    \item[$(*)$] Für alle $\psi\in FO(\tau)$ und alle Zuweisungen $\beta$ gilt $\mathfrak{A},\beta\models\psi[P_=/=]$ gdw. $\hat{\mathfrak{A}},\hat{\beta}\models\psi$.
  \end{itemize}
  Wegen $\mathfrak{A}\not\models\varphi[P_=/=]$ folgt $\hat{\mathfrak{A}}\not\models\varphi$, also ist $\varphi$ nicht gültig.
\end{description}
\qed
