\section{VL vom 09.~November 2010}

\subsection{Beweis des Resolutionslemmas}

\begin{itemize}
\item $V \models M \UNION \{C\} \rightarrow V \models  M$ trivial \\
\item Es gelte $V \models M$. Ferner sei $C = (C_1\ \{l\}) \UNION (C_2\ \{\overline{l}\})$ mit $C_1,C_2 \in M$. \\ Unterscheide 2 F"alle:
\begin{itemize}
    \item $V(l) = 1$ Wegen $V \models C_2$ dann auch $V \models C_2\ \{\overline{l}\}$ \\
    \item $V(l) = 0$ Wegen $V \models C_1$ dann auch $V \models C_1\ \{l\}$ \\
\end{itemize}
In beiden F"allen also $V \models C$ .
\end{itemize}

\subsubsection{Beispiel Resolution}

\begin{eqnarray*}
M(\varphi) &=& \{ \{x_1\}, \{\neg x_1, x_2\}, \{\neg x_2, x_3\}, \{\neg x_3\} \} = M \\
Res^0(M) &=& M \\
Res^1(M) &=& Res^0(M) \UNION \{ \{x_2\}, \{\neg x_1, x_3\}, \{\neg x_2\} \} \\
Res^2(M) &=& Res^1(M) \UNION \{ \{x_3\}, \{\neg x_1\}, \square \} \\
Res^3(M) &=& Res^2(M) = Res^*(M)
\end{eqnarray*}

\subsubsection{Resolutionssatz}

\paragraph[]{'<=' Korrektheit}
Da $\square \in Res^*(M)$, ist $Res^*(M)$ unerfüllbar. Es genügt also, zu zeigen, dass $Res^*(M) \equiv M$. Mittels Resolutionslemma und per Induktion über $i$ zeigt man leicht, dass $M \equiv Res^i(M)$ für alle $i \geq 0$. Da sich über den endlichg vielen Literalen in M nun endlich viele Klausen bilden lassen, ist $Res^*(M)$ endlich, also $Res^*(M) = Res^i(M)$ für ein $i >= 0$; damit $M \equiv Res^*(M)$.

\paragraph[]{'=>' Vollständigkeit}
Wir zeigen: $M \text{ unerfüllbar} \rightarrow \square \in M$.
per Induktion über $|Var(M)|$.

\paragraph[]{(IA)}
$Var(M) = \emptyset$. Dann $M=\emptyset$ oder $M=\{ \square \}$. Da $\emptyset$ erfüllbar, muss $M=\{ \square \}$ sein, also $\square \in M \SUBSET Res^*(M)$.

\paragraph[]{(IS)}
Wähle $x \in Var(M)$, konstruiere zwei Klauselmengen:
\begin{itemize}
\item $K^+ = \{C\ \{\neg x\} | x \not\in C \in M \}$ \\
\item $K^- = \{C\ \{x\} | \neg x \not\in C \in M \}$
\end{itemize}
Intuitiv entspricht $K^+$ dem Fall $V(x)=1$: alle Klauseln C mit $x \in C$ sind erfüllt und werden gestrichen, aus den verschiedenen Klauseln kann $\neg x$ gestrichen werden (wenn vorhanden), denn $\neg x$ kann die Klausel nicht wahr machen.

Wir zeigen:
\begin{itemize}
\item 1. $K^+ und K^-$ sind unerfüllbar \\
\item 2. $\square \in Res^*(M)$ oder $\{\neg x\} \in Res^*(M)$ \\
\item 3. $\square \in Res^*(M)$ oder $\{ x\} \in Res^*(M)$
\end{itemize}
Also:
\begin{itemize}
\item (1) Angenommen, $K^+$ ist erf"ullbar und $V \models K^+$. Erweitere V durch $V(x)=1$. Man pr"uft leicht, dass $V \models M$. \textbf{Widerspruch} Unerf"ullbarkeit $K^-$ analog.
\item (2) Weil $K^+$ unerf"ullbar liefert (IV) $\square \in Res^*(K^+)$. Also gibt es Klauseln $C_1,\dots,C_m$, so dass $C_m = \square$ und f"ur $1 \leq 0 \leq m$ gilt.
\begin{itemize}
    \item (a) $C_i \in K^+$ oder
    \item (b) $C_i$ ist Resolvente von $C_i, C_k$, f"ur gewisse $j, k < i$.
\begin{itemize}
        \item Fall 1: alle Klauseln $C_i$ der Form (a) sind auch in $M$ (in keiner der 'Originalklauseln' kam $\neg x$ vor). Dann pr"uft man leicht, dass $C_1,\dots,C_m \in Res^*(M)$, also $\square \in Res^*(M)$. \\
        \item Fall 2: F"ur mindestens ein $C_i$ der Form (a) ist $C_i \UNION \{\neg x\} \in M$. Wir erhalten durch Wiedereinf"ugen von $\neg x$ eine Folge von Klauseln:
$C^{'}_1,\dots,C^{'}_m \in Res^*(M)$.
die beweist, dass
\begin{verbatim}
C1  Ck      Ci \UNION {\neg x}    Ck \UNION \{\neg x\}
 \  /   =>                 \       /
  Ci                      Ci \UNION {\neg x}
\end{verbatim}
\end{itemize}
\end{itemize}
\item (3) Analog zu (2) unter Verwendung von $K^-$.
\end{itemize}
Aus (2) und (3) folgt, dass $\square \in Res^*(M)$ oder $\{x\}, \{\neg x\} \in Res^*(M)$. Mit
\begin{verbatim}
\neg x  x     
    \  /   
    \square
\end{verbatim}
das auch $\square \in Res^*(M)$. q.e.d.

\subsection{Einheitsresolvente}
\begin{verbatim}
            1                         2     3     4            5
{\neg x_1, \neg x_2, \neg x_3, x_4} {x_1} {x_2} {x_3} {\neg x_3, \neg x_4}
                    6
* 1/2 -> {\neg x_2, \neg x_3, x_4}
                7
* 6/3 -> {\neg x_3, x_4}
           8
* 7/4 -> {x_4}
           9
* 8/4 -> {\neg x_3}
           10
* 9/4 -> {\square}
\end{verbatim}



