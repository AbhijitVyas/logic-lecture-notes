\section{VL von 14.~Dezember 2010}

\subsection{Beispiel Herbrand-Struktur}

Sei $\tau = \set{c, f_1, f_2, P}$, wobei $c\in F^0(\tau)$ (Konstantensymbol)
$f_1,f_2 \in F^1(\tau)$ und $P \in R^2(\tau)$.

Für alle Herbrand-Strukturen für $\tau$ gilt:

\begin{align*}
  H &= \set{c, f_1(c), f_2(c), f_1(f_1(c)), f_1(f_2(c)), \dots} \\
  c^\Hfrak &= c \qquad f_1^\Hfrak(c) = f_1(c) \qquad f_2^\Hfrak(c) = f_2(c) \qquad \dots
\end{align*}

Die Interpretation von $P$ ist frei, z.B.

\begin{verbatim}
  Baum-/Graph-Struktur von Bernd/Carsten tikzen
\end{verbatim}

Es genügt, zu zeigen

\begin{itemize}
  \item[$(*)$] Für alle Sequenzen $\Gamma\IMPL\Delta$ und deren reduzierte Varianten
  $\Gamma'\IMPL\Delta'$ gilt: Jeder SK-Beweis für $\Gamma'\IMPL\Delta'$ kann in SK-Beweis
  für $\Gamma\IMPL\Delta$ gewandelt werden.
  
  Denn dann gilt:
  
  \begin{align*}
                            &\models\Gamma \IMPL\Delta \\
    \text{impliziert} \quad &\models\Gamma'\IMPL\Delta' && \text{(Äquivalenz)} \\
    \text{impliziert} \quad &\vdash \Gamma'\IMPL\Delta' && \text{(alle gültigen reduzierten Sequenzen ableitbar)} \\
    \text{impliziert} \quad &\vdash \Gamma \IMPL\Delta  && (*)
  \end{align*}
\end{itemize}

Beweisidee $(*)$:

Um SK-Beweis für $\Gamma\IMPL\Delta$ zu erhalten, folge derselben Strategie wie im
Beweis von $\Gamma'\IMPL\Delta'$. Jede auftretende Teilformel, die nicht reduziert
ist, wird mittels SK-Beweisen für

\begin{align*}
  \varphi \OR \psi &\EQUIV \NOT(\NOT\varphi \AND \NOT\psi) \\
  \forall x.\varphi(x) &\EQUIV \NOT\exists x.\NOT\varphi(x)
\end{align*}

\enquote{on the fly} in reduzierte Form gewandelt.

\subsection{??}

Wir konstruieren zwei Folgen:

\[ \Gamma_0 \subseteq \Gamma_1 \subseteq \dots \quad \text{und} \quad \Delta_0 \subseteq \Delta_1 \subseteq \dots \]

und setzen

\[ \Gamma* = \bigcup_{i\geq 0} \Gamma_i \quad \text{und} \quad \Delta* = \bigcup_{i\geq 0} \Delta_i\ \text{.} \]

Diese Konstruktion stellt sicher, dass

\begin{itemize}
  \item[$(*)$] $\Gamma_n \IMPL \Delta_n$ nicht ableitbar für alle $n\geq 0$.
\end{itemize}

Wir beginnen mit $\Gamma_0=\Gamma$ und $\Delta_0=\Delta$. Da $\tau$
abzählbar, gibt es Aufzählung

\[
  (\varphi_0, t_0), (\varphi_1, t_1), \dots
\]

so, dass jedes Paar $(\varphi,t)$ mit $\varphi$ Satz aus $FO(\tau)$ und $t$ Grundterm
aus $T(\tau)$ unendlich oft vorkommt:

\begin{verbatim}
  "Dovetailing" + Reset tikzen
  
  (\varphi_0, t_0),
  (\varphi_0, t_0),(\varphi_1, t_1),
  (\varphi_0, t_0),(\varphi_1, t_1),(\varphi_2, t_2),
  (\varphi_0, t_0),(\varphi_1, t_1),(\varphi_2, t_2),(\varphi_3, t_3),
  :
  .
\end{verbatim}

Für $n\geq 0$ konstruieren wir $\Gamma_{n+1}, \Delta_{n+1}$ wie folgt:

\begin{enumerate}
  \item $\varphi_n \in \NOT\psi$
  
  Wenn $\varphi_n \in \Gamma_n$, dann $\Gamma_{n+1}=\Gamma_n, \Delta_{n+1}=\Delta_n\cup\set{\psi}$.\\
  Wenn $\varphi_n \in \Delta_n$, dann $\Gamma_{n+1}=\Gamma_n\cup\set{\psi}, \Delta_{n+1}=\Delta_n$.\\
  (Sonst: $\Gamma_{n+1}=\Gamma_n, \Delta_{n+1}=\Delta_n$)
  
  Mit den Regeln ($\NOT\IMPL$) und ($\IMPL\NOT$) folgt, dass $\Gamma_{n+1}\IMPL\Delta_{n+1}$ nicht ableitbar.
  Exemplarisch erster Fall:
  
  Angenommen, $\Gamma_n\IMPL\Delta_n\cup\set{\psi}$ ist ableitbar. Dann mit ($\NOT\IMPL$)
  auch $\Gamma_n\cup\set{\NOT\psi}\IMPL\Delta_n=\Gamma_n\IMPL\Delta_n$. $\lightning$
  
  \item $\varphi_n = \psi_1 \AND \psi_2$
  
  Wenn $\varphi_n \in \Gamma_n$, dann $\Gamma_{n+1}=\Gamma_n\cup\set{\psi_1,\psi_2}, \Delta_{n+1}=\Delta_n$.
  Dann $\Gamma_{n+1}\IMPL\Delta_{n+1}$ nicht ableitbar wegen ($\AND\IMPL$).
  
  Wenn $\varphi_n\in\Delta_n$, dann ist

  \begin{enumerate}[label=(\roman*)]
    \item $\Gamma_n \in \Delta_n\cup\set{\psi_1}$ oder
    \item $\Gamma_n\IMPL\Delta_n\cup\set{psi_2}$ nicht ableitbar,
  \end{enumerate}

  denn sonst wäre ($\IMPL\AND$) auch $\Gamma_n\IMPL\Delta_n$ ableitbar.
  
  Setze $\Gamma_{n+1}=\Gamma_n$ und $\Delta_{n+1}=\Delta_n\cup\set{\psi_1}$.
  Wenn Fall (i) zutrifft.
  
  \item $\varphi_n = \exists x.\varphi(x)$
  
  Wenn $\varphi_n\in\Gamma_n$, dann wähle $c\in C$, welches nicht in
  $\Gamma_n\cup\Delta_n$ vorkommt. Setze $\Gamma_{n+1}=\Gamma_n\cup\set{\psi(c)},
  \Delta_{n+1}=\Delta_n$.
  
  Wenn $\varphi_n\in\Delta_n$, dann setze $\Gamma_{n+1}=\Gamma_n$ und
  $\Delta_{n+1}=\Delta_n\cup\set{\psi(t_n)}$.
\end{enumerate}

Mit den Regeln ($\exists\IMPL$) und ($\IMPL\exists$) folgt, dass
$\Gamma_{n+1}\IMPL\Delta_{n+1}$ nicht ableitbar.

Es bleibt zu überprüfen, dass $\Gamma^*,\Delta^*$ die Bedingungen 1--4
des Lemmas erfüllen. Exemplarisch 1+4:

\begin{enumerate}
  \item Wenn $\varphi\in\Gamma^*\cap\Delta^*$, dann gibt es $n\geq 0$ mit
  $\varphi\in\Gamma_n\cap\Delta_n$. Dann ist $\Gamma_n\IMPL\Delta_n$ ein
  Axiom und damit ableitbar, im Widerspruch zu (*).
  
  \item[4.] Wenn $\exists x.\varphi(x)\in\Gamma^*$, dann
  $\exists x.\psi(x)\in\Gamma_m$ für ein $m\geq 0$. Dann gibt es ein
  $n\geq m$ mit $\varphi_n=\exists x.\psi(x)$. Nach (3) gilt
  $\psi(c)\in\Gamma_{n+1}\subseteq\Gamma^*$ für eine Konstante $c$.
  
  Wenn $\exists x.\psi(x)\in\Delta^*$ und $t$ beliebiger Grundterm, dann
  gibt es $n\geq 0$ mit $\exists x.\psi(x)\in\Delta_n$,
  $\varphi_n=\exists x.\psi(x)$ und $t_n=t$. Also
  $\psi(t_n)\in\Delta_{n+1}\in\subseteq\Delta^*$.
\end{enumerate}
\qed

\subsubsection{Vollständig (kontrapositiv)}

Sei $\Gamma\IMPL\Delta$ nicht ableitbar, $\Gamma^*$ und $\Delta^*$ wie
im Lemma. Sei $\Afrak$ die kanonische Herbrand-Struktur von $\Gamma^*$,
also

\[
  R^\Afrak = \set{(t_1,\dots,t_n) | R(t_1,\dots,t_n)\in\Gamma^*}.
\]

Wir wollen zeigen, dass $\Afrak\models\ANDop\Gamma \AND \ANDop\NOT\Delta$,
denn dann ist $\Gamma\IMPL\Delta$ nicht gültig. Wegen $\Gamma\subseteq\Gamma^*$
und $\Delta\subseteq\Delta^*$ genügt es, zu zeigen, dass
$\Afrak\models\ANDop\Gamma^* \AND \ANDop \NOT\Delta^*$.

Dies folgt aus folgender Behauptung:

Für alle $\tau$-Sätze $\varphi$ gilt:

\begin{enumerate}
  \item $\varphi\in\Gamma^*$ impliziert $\Afrak\models\varphi$
  \item $\varphi\in\Delta^*$ impliziert $\Afrak\not\models\varphi$
\end{enumerate}
  
Beweis per Induktion über Ordnung \enquote{$c$} auf $FO(\tau)$, wobei
$\varphi<\psi$ wenn $\varphi$ aus Teilformel von $\psi$ hervorgeht, indem
freie Variable durch Grundterm ersetzt werden.

\begin{itemize}
  \item $\varphi$ ist ein Atom, also von der Form $R(t_1,\dots,t_n)$.
  Behauptung folgt direkt aus Definition von $\Afrak$ und aus Bedingung 1 des Lemmas.
  
  \item $\varphi = \NOT\psi$
  \begin{enumerate}
    \item $\varphi\in\Gamma^*\IMPL\psi\in\Delta^*$ (Eigenschaft 2 des Lemmas)\\
                            $\IMPL\Afrak\not\in\psi$ (IV)\\
                            $\IMPL\Afrak\models\varphi$ (Semantik)
    \item $\varphi\in\Delta^*\IMPL\psi\in\Gamma^*$ (Eigenschaft 2)\\
                            $\IMPL\Afrak\models\psi$ (IV)\\
                            $\IMPL\Afrak\not\in\varphi$
  \end{enumerate}

  \item $\varphi=\psi_1\AND\psi_2$      ähnlich zu vorigem Fall
  
  \item $\varphi=\exists x.\psi(x)$
  \begin{enumerate}
    \item $\varphi\in\Gamma^*\IMPL\psi[t]\in\Gamma^*$ für Grundterm $t$\\
                            $\IMPL\Afrak\models\psi[t]$ für Grundterm $t$ (IV)\\
                            $\IMPL\Afrak\varphi$
                          
    \item $\varphi\in\Delta^*\IMPL\psi[t]\in\Delta^*$ für Grundterm $t$\\
                            $\IMPL\Afrak\not\models\psi[t]$ für Grundterm $t$\\
                            $\IMPL\Afrak\not\models\varphi$ (da $A$ die Menge aller Grundterme ist).
  \end{enumerate}
\end{itemize}
\qed
